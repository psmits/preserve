\documentclass[11pt]{article}
\usepackage[sc]{mathpazo} %Like Palatino with extensive math support
\usepackage{fullpage}
\usepackage[authoryear,sectionbib,sort]{natbib}
\linespread{1.7}
\usepackage[utf8]{inputenc}
\usepackage{lineno}
\usepackage{titlesec}
\titleformat{\section}[block]{\Large\bfseries\filcenter}{\thesection}{1em}{}
\titleformat{\subsection}[block]{\Large\itshape\filcenter}{\thesubsection}{1em}{}
\titleformat{\subsubsection}[block]{\large\itshape}{\thesubsubsection}{1em}{}
\titleformat{\paragraph}[runin]{\itshape}{\theparagraph}{1em}{}[. ]\renewcommand{\refname}{Literature Cited}

%%%%%%%%%%%%%%%%%%%%%
% Line numbering
%%%%%%%%%%%%%%%%%%%%%
%\usepackage{lineno}
% Please use line numbering with your initial submission and
% subsequent revisions. After acceptance, please turn line numbering
% off by adding percent signs to the lines %\usepackage{lineno} and
% to %\linenumbers{} and %\modulolinenumbers[3] below.

\title{Template and Guidelines for Using \LaTeX{} in \textit{The~American~Naturalist} }

% This version of the LaTeX template was last updated on
% January 11, 2018.

%%%%%%%%%%%%%%%%%%%%%
% Authorship
%%%%%%%%%%%%%%%%%%%%%
% Please remove authorship information while your paper is under review,
% unless you wish to waive your anonymity under double-blind review. You
% will need to add this information back in to your final files after
% acceptance.

\author{Owen E. Cook$^{1,\ast}$ \\ 
Generic H. Collaborator$^{2,\dag}$ \\ 
Additional Q. Expert$^{3}$}

\date{}

\begin{document}

\maketitle

\noindent{} 1. University of Chicago, Chicago, Illinois 60637;

\noindent{} 2. University of Toronto, Toronto, Ontario M5S 1A5, Canada;

\noindent{} 3. Middle Eastern Technical University, Çankaya, Ankara 06800, Turkey.

\noindent{} $\ast$ Corresponding author; e-mail: amnat@uchicago.edu.

\noindent{} $\dag$ Deceased.

\bigskip

\textit{Manuscript elements}: Figure~1, figure~2, table~1, online appendices~A and B (including figure~A1 and figure~A2). Figure~2 is to print in color.

\bigskip

\textit{Keywords}: Examples, model, template, guidelines.

\bigskip

\textit{Manuscript type}: Article. %Or e-article, note, e-note, natural history miscellany, e-natural history miscellany, comment, reply, invited symposium, or countdown to 150.

\bigskip

\noindent{\footnotesize Prepared using the suggested \LaTeX{} template for \textit{Am.\ Nat.}}

%\linenumbers{}
%\modulolinenumbers[3]

\newpage{}

\section*{Abstract}

Lorem ipsum dolor sit amet, consectetur adipiscing elit. Sed non risus. Suspendisse lectus tortor, dignissim sit amet, adipiscing nec, ultricies sed, dolor. Cras elementum ultrices diam. 

\newpage{}

\section*{Introduction}

% The journal does not have numbered sections in the main portion of
% articles. Please refrain from using section references (à la
% section~\ref{section:CountingOwlEggs}), and refer to sections by name
% (e.g. section ``Counting Owl Eggs'').

The quick red fox jumps over the lazy brown dog. Furthermore, the quick brown fox jumps over the lazy red dog. In addition, the quick R\"{u}ppell's fox (\textit{Vulpes rueppellii}) jumps over the lazy golden retriever.

\section*{Methods}

Lorem ipsum (\citealt{Xiao2015}) ut velit mauris, egestas sed, gravida nec, ornare ut, mi. Aenean ut orci vel massa suscipit pulvinar. Nulla sollicitudin. Fusce varius, ligula non tempus aliquam, nunc turpis ullamcorper nibh, in tempus sapien eros vitae ligula. 

\subsection*{The quickness of the fox}

Lorem ipsum nulla facilisi, despite the findings of \citet{LemKapEx07}. Etiam semper, orci sit amet facilisis interdum, tellus nunc consequat erat, quis viverra nisi diam ut metus. Pellentesque cursus, sapien malesuada euismod iaculis, mauris purus interdum diam, vel vestibulum justo enim vitae tellus. Nunc interdum lorem sit amet diam volutpat tristique. Quisque pulvinar ac metus commodo lacinia (\citealt{Ing11,Xiao2015}).  

\subsubsection*{Third-order heading}

Usually two or three levels of heading will be all you need. Journal style even permits a fourth level in case you need it.

\paragraph*{Fourth-order heading}
The quick red fox jumps over the lazy brown dog in this paragraph as well.


\subsection*{The redness of the fox}

As \citet{Xiao2015} argued, phasellus porttitor eros et ante condimentum, eget facilisis orci condimentum. Nulla facilisi. Proin placerat elit blandit, euismod dolor nec, dapibus diam. Mauris posuere malesuada lacus, at elementum lacus auctor eu (fig~\ref{Fig:Jumps}A). 

\section*{Results}

Lorem ipsum dolor sit amet. Aenean pulvinar malesuada commodo (see \citealt{DavisEtAl2011}; table~\ref{Table:Okapi}). Sed aliquet mauris odio, in tristique dui egestas a. Etiam eu malesuada quam. Suspendisse tincidunt eu erat sit amet vulputate. Duis at arcu et nisl dictum mattis. Maecenas vel cursus ante. Cras eleifend elit nec velit sollicitudin fermentum in ac mauris. Pellentesque rutrum magna vel elit maximus hendrerit. 

\subsection*{The height of the jump}

Aenean eu pellentesque quam (fig.~\ref{Fig:OkapiHorn}). Nam pellentesque augue eu finibus lacinia. Nullam nec justo vitae odio imperdiet rhoncus vitae vitae quam. Pellentesque porttitor metus et lectus ornare, ac cursus urna efficitur (fig~\ref{Fig:Jumps}B). 

\subsection*{The laziness of the dog}

Sed sit amet pharetra nisi (video~\ref{VideoExample}, fig.~\ref{Fig:AnotherFigure}). Praesent quis dolor in dolor molestie cursus et ac nisi. Vestibulum ante purus, semper eget est vitae, vehicula ornare nisl. Morbi efficitur euismod enim, nec feugiat tellus cursus eget. Donec mauris nibh, volutpat vehicula viverra at, iaculis congue sem. Praesent eget erat rhoncus erat sollicitudin volutpat. 

If you have deposited data to Dryad, you should cite them somewhere in the main text (usually in the Methods or Results sections). A sentence like the following will do. All data are available in the Dryad Digital Repository (\citealt{CookEtAl2015}).

\section*{Discussion}

Nam pulvinar lorem at lorem ultrices, vel accumsan massa feugiat (\citealt{Ing11}). Proin tristique velit eget lacus iaculis, in pellentesque nulla varius. Phasellus sodales est odio, eu pulvinar magna pellentesque eu. Sed ut lobortis eros. Aliquam eget metus turpis. Sed et convallis lectus, id tincidunt enim. In porta nibh ut lacus feugiat, non consequat orci rhoncus. Morbi blandit at augue nec tempor. Sed fringilla ipsum ut justo viverra, ut euismod nisi gravida.

Curabitur non posuere augue, id suscipit orci. Nunc luctus accumsan aliquam. Cras egestas turpis vitae nisl vulputate interdum. Donec pellentesque libero egestas tortor pharetra laoreet. Phasellus facilisis auctor ligula, eu sollicitudin mi sagittis non.

\section*{Conclusion}

Duis pharetra enim at libero cursus, eu commodo mi vestibulum. Nullam eget velit nec lectus viverra sodales. Suspendisse egestas, eros at dictum tincidunt, mi orci laoreet libero, eget rutrum sapien arcu blandit odio.

%%%%%%%%%%%%%%%%%%%%%
% Acknowledgments
%%%%%%%%%%%%%%%%%%%%%
% You may wish to remove the Acknowledgments section while your paper 
% is under review (unless you wish to waive your anonymity under
% double-blind review) if the Acknowledgments reveal your identity.
% If you remove this section, you will need to add it back in to your
% final files after acceptance.

\section*{Acknowledgments}

OEC would like to thank the world. GHC is much indebted to the solar system. AQE was supported by a generous grant from the Milky Way (MW/01010/987654).

\newpage{}

\section*{Appendix A: Supplementary Figures}

% Please reset counters for the appendix (thus normally figure A1, 
% figure A2, table A1, etc.).

% In certain cases, it may be appropriate to have a PRINT appendix in
% addition to (or instead of) an online appendix. In this case, please 
% name the print appendix Appendix A, and any subsequent appendixes (if 
% there are any) should be named Online Appendix B, Online Appendix C,
% etc.

% Counters for each appendix should match the letter of that appendix.
% For example, tables in Appendix C should be numbered table C1, table C2,
% etc. This applies to tables, equations, and figures.

% It's better not to use the \appendix command, because we have some
% formatting peculiarities that \appendix conflicts with.

\renewcommand{\theequation}{A\arabic{equation}}
% redefine the command that creates the equation number.
\renewcommand{\thetable}{A\arabic{table}}
\setcounter{equation}{0}  % reset counter 
\setcounter{figure}{0}
\setcounter{table}{0}

\subsection*{Fox--dog encounters through the ages}

The quick red fox jumps over the lazy brown dog. The quick red fox has always jumped over the lazy brown dog. The quick red fox began jumping over the lazy brown dog in the 19th century and has never ceased from so jumping, as we shall see in figure~\ref{Fig:Jumps}. But there can be surprises (figure~\ref{Fig:JumpsOk}).

If the order and location of figures is not otherwise clear, feel free to include explanatory dummy text like this:

[Figure A1 goes here.]

[Figure A2 goes here.]

\subsection*{Further insights}

Tables in the appendices can appear in the appendix text (see table~\ref{Table:Rivers} for an example), unlike appendix figure legends which should be grouped at the end of the document together with the other figure legends.

\begin{table}[h]
\caption{Various rivers, cities, and animals}
\label{Table:Rivers}
\centering
\begin{tabular}{lll}\hline
River        & City        & Animal            \\ \hline
Chicago      & Chicago     & Raccoon           \\
Des Plaines  & Joliet      & Coyote            \\
Illinois     & Peoria      & Cardinal          \\
Kankakee     & Bourbonnais & White-tailed deer \\
Mississippi  & Galena      & Bald eagle        \\ \hline
\end{tabular}
\bigskip{}
\\
{\footnotesize Note: See table~\ref{Table:Okapi} below for further table formatting hints.}
\end{table}

Lorem ipsum dolor sit amet, as we have seen in figures~\ref{Fig:Jumps} and \ref{Fig:JumpsOk}.

\newpage{}

\section*{Appendix B: Additional Methods}

\subsection*{Measuring the height of fox jumps without a meterstick}

Pellentesque ac nibh placerat, luctus lectus non, elementum mauris. 
Morbi odio velit, eleifend ut hendrerit vitae, consequat sit amet 
nulla. Pellentesque porttitor vitae nisl quis tempus. Pellentesque 
habitant morbi tristique senectus et netus et malesuada fames ac 
turpis egestas. Praesent ut nisi odio. Vivamus vel lorem gravida 
odio molestie volutpat condimentum et arcu. 

\begin{equation}
{ \frac{1}{N_k-1} \sum \limits_{t=1}^{N_k} (M_{tjk} - \bar{M}_{jk})^2}
\end{equation}

\subsection*{Quantifying the brownness of the dog}

Pellentesque eu nulla odio (\citealt{Xiao2015,CookEtAl2015}). Nulla aliquam porta metus, quis malesuada orci faucibus quis. Suspendisse nunc magna, tristique sit amet sollicitudin nec, elementum et lacus. Sed vitae elementum mi. In hac habitasse platea dictumst. Etiam eu tortor elit. Sed ac tortor purus. Aliquam volutpat, odio sit amet posuere pretium, dolor ex interdum ante, sed luctus quam eros ac nulla. 

\begin{equation}
{ (\sum \limits_{p=1}^P {n_{sp}})^{-1}\sum \limits_{p=1}^P {n_{sp}Q_{p}}}
\end{equation}

\newpage{}

%%%%%%%%%%%%%%%%%%%%%
% Bibliography
%%%%%%%%%%%%%%%%%%%%%
% You can either type your references following the examples below, or
% compile your BiBTeX database and paste the contents of your .bbl file
% here. The amnatnat.bst style file should work for this---but please
% let us know if you run into any hitches with it!
% The list below includes sample journal articles, book chapters, and
% Dryad references.

\begin{thebibliography}{}

\bibitem[{Cook et~al.(2015)Cook, Collaborator, and Expert}]{CookEtAl2015}
Cook, O.~E., G.~H. Collaborator, and A.~Q. Expert. 2015.
\newblock Data from: Template and guidelines for using \LaTeX{} in \textit{The American Naturalist}.
\newblock American Naturalist, Dryad Digital Repository, http://dx.doi.org/10.5061/dryad.XYZAB.

\bibitem[{Davis et~al.(2011)Davis, Brakora, and Lee}]{DavisEtAl2011}
Davis, E.~B., K.~A. Brakora, and A.~H. Lee. 2011.
\newblock Evolution of ruminant headgear: a review.
\newblock Proceedings of the Royal Society B 278:2857--2865.

\bibitem[{Inglis et~al.(2011)Inglis, Roberts, Gardner, and Buckling}]{Ing11}
Inglis, R.~F., P.~G. Roberts, A.~Gardner, and A.~Buckling. 2011.
\newblock Spite and the scale of competition in \textit{Pseudomonas
  aeruginosa}.
\newblock American Naturalist 178:276--285.

\bibitem[{Lemod\`{e}le et~al.(2007)Lemodele, Kapitelschreiber, and Exemplar}]{LemKapEx07}
Lemod\`{e}le, P.-Q., A.~B. Kapitelschreiber, and C.~D.~E. Exemplar. 2007.
\newblock An exemplary instance of chapters in books.
\newblock Pages 231--245 \emph{in} J.-P. \'{E}crivain and M.~A. Term\'{e}szettud\'{o}s, eds. Inspiring Instances of Brilliant Writing. Truth Pudding Press, Fond du Lac, WI.

\bibitem[{Xiao et~al.(2015)Xiao, McGlinn, and White}]{Xiao2015}
Xiao, X., D.~J. McGlinn, and E.~P. White. 2015.
\newblock A strong test of the maximum entropy theory of ecology.
\newblock American Naturalist 185:E705--E80.

\end{thebibliography}

\newpage{}

\section*{Tables}
\renewcommand{\thetable}{\arabic{table}}
\setcounter{table}{0}

\begin{table}[h]
\caption{Animals in various cities with equations}
\label{Table:Okapi}
\centering
\begin{tabular}{llc}\hline
Animal    & City         & Equation \\ \hline
Dog       & Springfield  & $x+y=z$ \\
Fox       & Indianapolis & $2x+2y=2z$ \\
Okapi$^a$ & Chicago      & $x-y<z$ \\
Badger    & Madison      & $x+2y>z$ \\ \hline
\end{tabular}
\bigskip{}
\\
{\footnotesize Note: Table titles should be short. Further details should go in a `notes' area after the tabular environment, like this. $^a$ Okapis are not native to Chicago, but they are to be met with in both of the major Chicagoland zoos.}
\end{table}

\newpage{}

\section*{Figure legends}

\begin{figure}[h!]
%\includegraphics{horn-of-okapi}
\caption{Figure legends can be longer than the titles of tables. However, they should not be excessively long.}
\label{Fig:OkapiHorn}
\end{figure}


%%%%%%%%%%%%%%%%%%%%%
% Videos
%%%%%%%%%%%%%%%%%%%%%
% If you have videos, journal style for them is similar to that for
% figures. You'll want to include a still image (such as a JPEG)
% to give your readers a preview of what the video looks like.

%%%%% Include the text below if you have videos

\renewcommand{\figurename}{Video} 
\setcounter{figure}{0}
% Thanks to Flo Debarre for the pro tip of putting
% \renewcommand{\figurename}{Video} before the Video legend and
% \renewcommand{\figurename}{Figure} after it!

\begin{figure}[h!]
%\includegraphics{VideoScreengrab.jpg}
\caption{Video legends can follow the same principles as figure legends. Counters should be set and reset so that videos and figures are enumerated separately.}
\label{VideoExample}
\end{figure}

\renewcommand{\figurename}{Figure}
\setcounter{figure}{1}

%%%%% Include the above if you have videos


\begin{figure}[h!]
%\includegraphics{elegance}
\caption{In this way, figure legends can be listed at the end of the document, with references that work, even though the graphic itself should be included for final files after acceptance. Instead, upload the relevant figure files separately to Editorial Manager; Editorial Manager should insert them at the end of the PDF automatically.}
\label{Fig:AnotherFigure}
\end{figure}

\subsection*{Online figure legends}

\renewcommand{\thefigure}{A\arabic{figure}}
\setcounter{figure}{0}

\begin{figure}[h!]
%\includegraphics{jumps20m}
\caption{\textit{A}, the quick red fox proceeding to jump 20~m straight into the air over not one, but several lazy dogs. \textit{B}, the quick red fox landing gracefully despite the skepticism of naysayers.}
\label{Fig:Jumps}
\end{figure}

\begin{figure}[h!]
%\includegraphics{jumps20m}
\caption{The quicker the red fox jumps, the likelier it is to land near an okapi. For further details, see \citet{LemKapEx07}.}
\label{Fig:JumpsOk}
\end{figure}

\renewcommand{\thefigure}{B\arabic{figure}}
\setcounter{figure}{0}

\end{document}
