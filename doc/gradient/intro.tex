\documentclass[12pt,letterpaper]{article}

\usepackage{amsmath, amsthm}
\usepackage{microtype, parskip}
\usepackage[comma,numbers,sort&compress]{natbib}
\usepackage{lineno}
\usepackage{docmute}
\usepackage{caption, subcaption, multirow, morefloats, rotating}
\usepackage{wrapfig}

\frenchspacing

\captionsetup[subfigure]{position = top, labelfont = bf, textfont = normalfont, singlelinecheck = off, justification = raggedright}

\begin{document}
\section{Introduction}

% latitudinal diversity gradients
%   hypotheses as to why, in terms of diversification
% diversity dependence
%   the idea that there are ``ecological limits'' on biodiversity
%     limiting similarity/competitive exclusion
%       similarity in ecology/niche due to high phylogenetic similarity 
%     resources
%     space
%   see Harmon vs Rabosky debate in AmNat
%   seen as sign of ``biological interaction''
%     probably more complicated than that, as nothing is A or B but rather A and B.


% Phanerozoic brachiopods
%   gain, loss, gain, etc. LDG




\end{document}
