\documentclass[12pt,letterpaper]{article}

\usepackage{amsmath, amsthm}
\usepackage{microtype, parskip}
\usepackage[comma,numbers,sort&compress]{natbib}
\usepackage{lineno}
\usepackage{docmute}
\usepackage{caption, subcaption, multirow, morefloats, rotating}
\usepackage{wrapfig}

\frenchspacing

\begin{document}
\section{Materials and Methods}

\subsection{Occurrence information}

\subsection{Diversity model}

This model is based on the Jolly-Seber model for capture-mark-recapture data CITATION. There are three major parameters to this model: survival probability \(\phi\), recruitment probability \(\gamma\), and sampling probability \(p\). Implicit in this model is also removal probabilty \(1 - \phi\).

\(\phi_{t - 1}\) is the probability of surving to time \(t\) from \(t - 1\) given presence at time \(t - 1\). Survival probability is a function of both extinction and extirpation.

\(\gamma_{t}\) is the probability of first appearing at time \(t\) given absence at time \(t - 1\). Recruitment is a function of both origination and colonization.

\(p_{t}\) is the probability of observing \(y_{i}\) at time \(t\) if it is present.


\begin{equation}
  \begin{aligned}
    y_{i,t} &\sim \mathrm{Bernoulli}(p_{t} z_{i, t}) \\
    z_{i, t} &\sim
    \begin{cases}
      \mathrm{Bernoulli}(\gamma_{1}) & \text{if } t = 1, \text{and} \\
      \mathrm{Bernoulli}(z_{i, t} \phi_{t - 1} + (1 - z_{i, t}) \gamma_{t}) & \text{if } t > 1. \\
    \end{cases}
  \end{aligned}
\end{equation}


\end{document}
