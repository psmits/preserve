\documentclass[12pt,letterpaper]{article}

\usepackage{amsmath, amsthm}
\usepackage{microtype, parskip}
\usepackage[comma,numbers,sort&compress]{natbib}
\usepackage{lineno}
\usepackage{docmute}
\usepackage{caption, subcaption, multirow, morefloats, rotating}
\usepackage{wrapfig}

\frenchspacing

\captionsetup[subfigure]{position = top, labelfont = bf, textfont = normalfont, singlelinecheck = off, justification = raggedright}

\begin{document}
\section{Discussion}

% hypotheses
%   Jablonski1986 hypothesis
My results demonstrate that both the effects of geographic range and the peakedness/concavity of environmental preference are both negatively correlated with baseline extinction risk, meaning that as baseline extinction risk increases the effect sizes of both these traits are expected to increase (Fig. \ref{fig:corr}). This result supports neither of the two proposed macroevolutionary mechanisms for how biological traits should correlate with extinction risk. The observed correlation between the two effects as well as between the effects and baseline extinction risk instead implies that as baseline extinction risk increases, the strength of the total selection gradient on biological traits (except body size) increases. This manifests as greater differences in extinction risk for each unit difference in the biological covariates during periods of high extinction risk, while a relatively flatter selection gradient during periods of low extinction risk.

For the approximately 233 My period analyzed there is an approximate 75\% posterior probability that brachiopod genera with intermediate environmental preferences are expected to have a lower extinction risk than either end members. However, the over all curvature of \(f(v_{i})\) is not very peaked meaning that when averaged over the entire Phanerozoic this relationship may not lead to large differences in extinction risk (Fig. \ref{fig:env_mean}). Note that the duration of the period analyzed is approximately four times then length of the Cenozoic (e.g. time since the extinction of the non-avian dinosaurs). This result gives weak support for the universality of the hypothesis that environmental generalists have greater survival than environmental specialists \citep{Simpson1944,Liow2004a,Liow2007b,Nurnberg2013a,Nurnberg2015}.

The posterior variance in the estimate of overall \(f(v_{i})\) reflects the large between cohort variance in cohort specific estimates of \(f(v_{i})\) (Fig. \ref{fig:env_cohort}). Given that there is only a 75\% posterior probability that the expected overall estimate of \(f(v_{i})\) is concave down, it is not surprising that there are some stages where the estimated relationship is in fact the reverse of the prior expectation. Additionally, some of those same stages where \(f(v_{i})\) does not resemble the prior expectation of a concave down nonlinear relation are instead is highly skewed and effectively linear (Fig. \ref{fig:env_cohort}). These results demonstrate that, while the group-level estimate may only weakly support one hypothesis, the cohort-level estimates may exhibit very different characteristics.These results are also consistent with aspects of \citet{Miller2009a} who found that the effect of environmental preference on extinction risk was quite variable and without obvious patterning during times of background extinction.


%   Miller2009a results
There are two mass extinction events that are captured within the time frame considered here: the Ordovician-Silurian and the Frasnian-Famennian. The cohorts bracketing these events are worth considering in more detail.

% write about the mass extinctions \ref{fig:env_cohort}

The proposed mechanism for the end Ordovician mass extinction is a decrease in sea level and the draining of epicontinental seas due to protracted glaciation \citep{Sheehan2001b,Johnson1974}. My results are broadly consistent with this scenario with both epicontinental and open-ocean specialists having a much lower expected duration than intermediate taxa (Fig. \ref{fig:env_cohort}). All of the stages between the Darriwillian and the Llandovery, except the Hirnantian, have a high probability (90+\%) that \(f(v)\) is concave down. The pattern for the Darriwillian, which proceeds the supposed start of Ordovician glacial activity, demonstrates that taxa tend to favor open-ocean environments are expected to have a greater duration than either intermediate of epicontinental specialists, in decreasing order.

For nearly the entire Devonian estimates of \(f(v)\) indicate that one of the environmental end members is favored over the other end member of intermediate preference (Fig. \ref{fig:env_cohort}). This is consistent with the predictions of \citet{Miller2009a}. For almost the entirely the Givetian though the end of the Devonian and into the Vis\'{e}an, I find that epicontinental favoring taxa are expected to have a greater duration than either intermediate or open-ocean specialists. Additionally, for nearly the entire Devonian and through to the Visean, the cohort-specific estimates of \(f(v)\) are concave-up. This is the opposite pattern than what is expected (Fig. \ref{fig:env_mean}). This result, however, seems to reflect the intensity of the seemingly nearly-linear difference in expected duration across the range of \(v)\) as opposed to an inversion of the weakly expected curvilinear pattern.

% defense
%   species:genus?
%   difficulty towards tails, but that's to be expected
%     this model is about expectations, not tails/extreme events
%     this model is ok for the main part of the data
%     though, of course, this model has a long way to go (all models are false)
Of concern is the use of genera as the unit of the study and how to exactly interpret the effects of the biological traits. For example, if any of the traits analyzed here are associated with increases in speciation rates, this might ``artificially'' increase the duration of genera through self-renewal \citep{Raup1991b,Raup1994}. This could lead to a trait appearing to decrease generic level extinction risk by increasing species level origination rate instead of decreasing species level extinction risk. However, given the nature of the brachiopod fossil record and the difficulty of identifying individual specimens to the species level, there is no simple solution to decreasing this uncertainty in the interpretations of how the biological traits studied here actually affect extinction risk.

% future direction
This model could be improved through either increasing the number of analyzed taxon traits, expanding the hierarchical structure of the model to include other major taxonomic groups of interest, and the inclusion of explicit phylogenetic relationships between the taxa in the model as an additional hierarchical effect.

%   other measures of ecology? affixing strategy a la Alexander1977
An example taxon trait that may be of particular interest is the affixing strategy or method of interaction with the substrate of the taxon. This trait has been found to be related to brachiopod survival \citep{Alexander1977} so its inclusion may be of particular interest.

%   comparison with other major groups in hierarchical model
It is theoretically possible to expand this model to allow for comparisons within and between major taxonomic groups. This approach would better constrain the brachiopod estimates while also allowing for estimation of similarities and differences in cross-taxonomic patterns. The major issue surrounding this particular expansion involves finding an similarly well sampled taxonomic group that is present during the Paleozoic. Example groups include Crinoidea, Ostracoda, and other ``Paleozoic'' groups \citep{SepkoskiJr.1981a}.

%   integration of phylogenetic information/taxonomic component
%     taxon traits are more than likely heritable
%     what aspect of variation is explained just by 
%     see Smits Submitted
Taxon traits like environmental preference or geographic range \citep{Jablonski1987,Hunt2005b} are most likely heritable, at least phylogenetically \citep{Lynch1991,Housworth2004}. Without phylogenetic context, this analysis assumes that differences in extinction risk between taxa are independent of those taxa's shared evolutionary history \citep{Felsenstein1985b}. In contrast, the origination cohorts only capture shared temporal context. The inclusion of phylogenetic context as an addition individual level hierarchical structure independent of origination cohort would allow for determining how much of the observed variability is due to shared evolutionary history versus actual differences associated with these taxonomic traits. 

% concluding statements
In summary, patterns of Paleozoic brachiopod survival were analyzed using a fully Bayesian hierarchical survival modelling approach while also eschewing the traditional separation between background and mass extinction. I modeled both the overall mean effect of biological covariates on extinction risk while also modeling the correlation between cohort-specific estimates of covariate effects. I find that as baseline extinction risk increases, the strength of the selection gradient on biological traits (except body size) increases. This manifests as greater differences in extinction risk for each unit difference in the biological covariates during periods of high extinction risk, while a much flatter total selection gradient during periods of low extinction risk. I also find some support for ``survival of the unspecialized'' \citep{Simpson1944,Liow2004a,Liow2007b,Nurnberg2013a,Nurnberg2015} as a general characterization of the effect of environmental preference on extinction risk (Fig. \ref{fig:env_mean}), though there is heterogeneity between origination cohorts with most periods of time conforming to this hypothesis (Fig. \ref{fig:env_cohort}). 

\end{document}
