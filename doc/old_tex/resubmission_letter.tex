\documentclass{letter}
\usepackage{microtype}
\usepackage{letterbib}
\usepackage{natbib}
\frenchspacing

\signature{Peter D Smits}
\address{Committee on Evolutionary Biology \\ University of Chicago \\
1025 E. 57th Street \\ Culver Hall 402 \\ Chicago, IL 60637 \\
psmits@uchicago.edu}

\begin{document}
\begin{letter}{Editor \\ \textit{Evolution}}
  \opening{Dear Editor,}

  Please find enclosed the manuscript entitled ``How macroecology affects macroevolution: the interplay between extinction intensity and trait-dependent extinction in brachiopods'' which I previously submitted to this journal as ``The interplay between extinction intensity and selectivity: correlation in trait effects on taxonomic survival'' (15-0774).
  
  Extinction is one half of the diversification process \cite{Raup1994,Stanley1979,Stanley1975}, second only to speciation or origination. Extinction can also be though of as the ultimate manifestation of selection as a taxon with a beneficial trait should persist for longer on average than a taxon without that beneficial trait \cite{Rabosky2010b,Jablonski2008a,Raup1994,Stanley1975}. In the enclosed study I analyze how trait-dependent extinction rates in brachiopods change over time in relation to extinction intensity, time of origination, and the effects of other traits themselves.

  While estimation of both trait-dependent speciation and extinction rates from phylogenies of extant taxa is common \cite{Maddison2007,Fitzjohn2010,Goldberg2011a,Goldberg2005,Rabosky2013,Stadler2013b,Stadler2011a,Stadler2013a}, there are two major ways to estimate trait-dependent extinction: analysis of phylogenies, and analysis of the fossil record. These two approaches, phylogenetic comparative and paleobiological, are complementary and intertwined in the field of macroevolution \cite{Rabosky2010b,Jablonski2008a,Hunt2014a}. In the case of extinction, analysis of the fossil record has the distinct advantage of extinction being observable; this means that extinction rate is possible to estimate \cite{Rabosky2010a,Quental2009,Liow2010a}. The approach I used in my analysis is thus complementary to the analysis of trait-dependent extinction based on a phylogeny.

  Jablonski \cite{Jablonski1986} observed that for bivalves at the end Cretaceous mass extinction event, the effects of some biological traits on taxonomic survival decreased. However, this pattern was not the case for the effect of geographic range on survival \cite{Jablonski1986,Payne2007}. There are multiple possible macroevolutionary mechanisms which may underlie this pattern: the effect of geographic range on survival remains constant and those of other biological traits decrease, the effect of geographic range on survival increases and those of other biological traits stay constant, or the effects of all traits decrease but by different degrees.

  The emergent traits traits I analyzed in relation to their patterns of Paleozoic brachiopod genus durations were geographic range, affinity for epicontinental seas versus open ocean environments, and body size. Additionally, I estimated the effects of environmental generalization versus specialization on taxonomic survival by allowing environmental preference to have a nonlinear effect on duration. I also estimated the effect of fossil preservation on observed duration in order to constrain the interpretations of my results. 
  
  I found that the cohort-specific effects of geographic range and environmental preference are negatively correlated with baseline extinction intensity. Additionally, I found support for greater survival of environmental generalists versus specialists in all origination cohorts. These results support the conclusion that for Paleozoic brachiopods, as extinction intensity increases overall extinction selectivity increases.

  All data and code necessary to duplicate this analysis are available as an archived Zenodo repository DOI 10.5281/zenodo.46928.

  The reviewers of the previous manuscript were Paul Harnik (paul.harnik@fandm.edu, Franklin and Marshall College), and Lee Hsiang Liow (l.h.liow@ibv.uio.no, University of Oslo). Both of their reviews, along with that of the Associate Editor Alistair Evans, have greatly improved both the analysis and the manuscript. Here I will respond to the commentary by all three of them.
  
  Thank you for considering my work. Please send all correspondence regarding this manuscript to me via my email address (psmits@uchicago.edu).

  \encl{Manuscript, figures, tables.}

  \clearpage

  \section{Assistant Editor comments}
  \textbf{Minor comments: \\
  L47: if there are more than two traits, then ‘between’ should be ‘among’. \\
  L58: change to ‘trait-based’. \\
  L105: change to ‘long-standing’. \\
  L109: add period. \\
  L148: ‘drawn towards to over all mean’ is unclear (cf. L191). \\
  L196: change to ‘covariate’. \\
  L210: change to ‘age-dependent’. \\
  L325-326 and L326-327: change to ‘between-cohort variation’ (same as L323). Not sure how ‘between origination cohort’ should be hyphenated. \\
  L356: add period. \\
  L357: change ‘between’ to ‘among’. \\
  L359: change to ‘cohort-specific’ (and L360). \\
  L381: change to ‘between-cohort’. \\
  L411: change to ‘overall’. \\
  L414: change ‘then’ to ‘the’. \\
  L419-420: change to ‘between-cohort’ and ‘cohort-specific’. \\
  L428: add space after period. \\
  L434: perhaps ‘spanning’ rather than ‘bracketing’. \\
  L443: add ‘that’ after ‘taxa’. \\
  L446: add comma after ‘Devonian’. \\
  L448: ‘the entirely the Givetian’ is unclear. \\
  L449: change to ‘epicontinental-favoring’. \\
  L469: move ‘analyzed’ after ‘traits’. \\
  L481: change ‘an’ to ‘a’. \\
  L483: delete ‘Jr.’. \\
  L502: add ‘of’ after ‘effect’. \\
  L507: add comma after ‘extinction’. \\
  Table 1 caption: change to ‘intercept terms of the effects’. Capitalise ‘Equation 2’. \\
Fig. 5 caption: change from ‘lower left’ to ‘lower right’. There do not appear to be any numbers that are highlighted in red.}

  All minor comments from the Associate editor were incorporated into the manuscript where appropriate.
 
  \section{Reviewer \#1: Paul Harnik}
  \textbf{My main analytical question concerns the results related to environmental preference. Whereas the overall relationship between environmental preference and extinction risk is “concave down” (Fig. 2 and associated text) with genera that exhibit no affinity at lower risk than genera that prefer either epicontinental seas or open-ocean settings, the results for individual cohorts (Fig. 5) indicate that this was not always the case with some intervals (e.g., Wordian) exhibiting seemingly linear relationships and others “concave up” relationships (e.g., Visean). On page 19 (lines 452-455), you conclude that the “concave-up” patterns observed from the Givetian through Visean do not result from an inversion of the general Paleozoic relationship but rather “seemingly nearly-linear” relationships that due to the functional form of f(v) appear curvilinear. This explanation is also provided on pg. 16 (lines 374-5). Why would this same explanation not apply to the “concave down” patterns observed in other intervals? In the current manuscript, “concave up” patterns seem to be explained away post hoc whereas “concave down” patterns are seemingly interpreted as accurate renderings of non-linear extinction selectivity patterns. Why? This analysis might also be strengthened by quantifying the support for linear versus curvilinear relationships between environmental preference and extinction risk. This is particularly relevant in light of the discussion of the interactions between extinction intensity and selectivity with respect to range size and environmental preference on page 21.}

  The measure of environmental preference has changed, which has led to very different results (e.g. Figure 5). I also now estimate the probability of a concave-up relationship at 397-399.

  \textbf{ln 21: replace “analysis” with “analytical”}

  Done.

  \textbf{ln 27-28: consider rephrasing as the “correlation between the effects of geographic range and the non-linear aspect of environmental preference” will be unclear to anyone reading the abstract who has not already read the paper}

  Done.

  \textbf{ln 30-32: you write that “overall extinction selectivity” decreases with increasing extinction intensity but that isn’t the case for geographic range size (Fig. 3) so this statement does not hold for all selectivity patterns. Rephrase.}

  Done.

  \textbf{ln 41: With respect to Jablonski 1986 it may be worth noting that range size selectivity at the species (but not genus) level did weaken.}


  \textbf{ln 41-53: Here, and elsewhere in the manuscript, I would consider revising the text to be more accessible and direct. E.g., rather than “transferable to a continuous variation framework” emphasize that background and mass extinction rates occur on a continuum and that testing Jablonski’s hypothesis does not require discretizing extinctions.}

  Done.

  \textbf{ln 77-82: This paragraph is on the ‘importance’ of geographic range selectivity, yet its relative importance in determining extinction risk cannot be established in univariate analyses. I would consider restructuring and bolstering your argument by citing multivariate analyses (e.g., Jablonski 2008 PNAS; Crampton et al. 2010 Paleobiology; Harnik et al 2012 Proc B; Finnegan et al. 2012 PNAS, and others). Also, in addition to Foote and Miller (2013), Myers et al. 2013 Paleobiology did not observe range size selectivity among Cretaceous invertebrates.}

  Done.

  \textbf{Pgs. 3-4: Discussion of the effects of environmental preference on extinction risk focuses exclusively on the effects of sea level change on habitat area, yet other potentially important environmental characteristics such as ocean circulation and dysoxia differ between these settings which might be worth noting (e.g., see Peters 2007 Problem with the Paleozoic. Paleobiology).}

  The discussion of environmental preference in the Introduction has been expanded to discuss the mechanisms of differential extinction; see lines 119-131.


  \textbf{ln 95-101: Indicate to readers how current study differs from Miller and Foote (2013) in spatial, temporal, and taxonomic focus.}

  Done.

  \textbf{ln 110-114: Why you chose to consider body size in analyses of selectivity needs to be more clearly stated. In some groups (e.g., subclades of bivalves, see Harnik 2011 PNAS), the association between body size and duration may exceed that of geographic range. Worth citing here and/or in the discussion Harnik et al. 2014 Paleobiology which considered the effects of body size and geographic range size on extinction selectivity in a clade of brachiopods during the mid-Paleozoic and contrasted the results of taxonomic versus phylogenetically-informed analyses.}

  Done.

  \textbf{ln 131-132: Unclear why you assert that species dynamics are inherently of “greater interest”. I would consider rephrasing.}

  Done.

  \textbf{ln 135 for a recent comparison of species versus genus level patterns see Hoehn et al. Methods in Ecology and Evolution, available early online this fall.}

  Done.

  \textbf{ln 147-148: minor grammatical issues here and elsewhere in the manuscript.}

  Fixed.

  \textbf{ln 160-164: Not all readers of Evolution will know why you normalized your geographic range data the way that you, and many other paleontologists, have. Add 1-2 sentences about temporal variation in the quality of the fossil record.}

  Done.

  \textbf{ln 171: Indicate how these data transformations improve interpretability rather than assuming readers know this already. Also, were environmental affinities transformed?}

  Done.

  \textbf{pg. 8: Why did you choose to calculate environmental affinities integrated over the entire duration of each taxon whereas geographic range is the mean occupancy in each stage in which the genus was extant and sampled? By doing so you are assuming that environmental affinities are conserved which may or may not be warranted (e.g., see Hopkins et al. 2013 Ecology Letters in which ~25\% of genera shift substrate affinities over time).}

  Addressed in Discussion; see lines 531-546. % CHANGE

  \textbf{pg. 9-10: You “define” the models here yet several listed terms are not defined in the main text. I would consider placing more of this in the Appendix or defining more thoroughly in the main text.}

  The methods section has been restructured so that there is no longer a need for the appendices.

  \textbf{ln 284: Is there a measure of goodness-of-fit that is relevant to add here?}

  I have added an additional visual posterior predictive check (Figure 2) in order to better discuss the quality of model fit. These posterior predictive checks are the ``measures'' of goodness-of-fit.

  \textbf{ln 296-298: Comparable results for body size selectivity in Harnik et al. 2014 Paleobiology; i.e., negative relationship between size and duration that was not statistically different from zero.}

  Included in Introduction of the manuscript.

  \textbf{ln 386 \& ln 393 \& ln 464: choose a word rather than using “/” and leaving it to the reader}

  The text has been altered.

  \textbf{ln 416-418: This statement may be overly broad given that affinities for open-ocean versus epicontinental settings do not subsume other potentially relevant broad-scale axes of environmental breadth such as substrate, thermal tolerance, etc. Yet other analyses (e.g., Harnik et al. 2012 Proc B) have found that habitat breadth doesn’t consistently (nor substantially) contribute to extinction risk after geographic range size is accounted for.}

  The Discussion has been adjusted to better couch my results. See lines 516-530. % CHANGE

  \textbf{ln 437-445: Unclear why you expect open-ocean specialists to have higher extinction risk during past glacial intervals given statements about habitat persistence made in the Intro and the lack of extinction among shallow marine inverts during recent Pleistocene glacial-interglacial cycles.}

  I expect this result because of Miller and Foote's results \cite{Miller2009a}; this is mentioned more clearly now in the introduction. 

  \textbf{ln 484-497: In discussing potential confounding effects of phylogeny I think that it is worth noting that phylogenetic comparative analyses of extinction selectivity in Devonian terebratulides (Harnik et al. 2014 Paleobio) showed phylogenetic signal in both traits and durations yet estimated selectivity patterns did not differ significantly from taxonomic analyses.}

  Included in the longer discussion of phylogeny comparative approaches; see lines 592-613. % CHANGE

  \textbf{References: Formatting needs work.}

  Done.

  \textbf{Appendix C: worth noting that here, as in many previous analyses of selectivity, you are assuming that observed first occurrences and last occurrences accurately reflect the true timing of first and last occurrence. This aspect of uncertainty is not incorporated into the current study.}

  The issue of sampling and its effect on duration is discussed now in the introduction. Additionally, the effect of sampling is now also estimated as part of the model; its effect is smaller than the effect of geographic range, and both environmental preference effects.


  \section{Reviewer \#2, Lee Hsiang Liow}
  \textbf{I am curious, though, why “intensity” is in the title. Is it being used interchangeably with “extinction risk”? In the literature, I think extinction intensity often refers to time intervals while extinction risk refers to taxa (and selectivity to taxa with certain traits).}

  Changes in extinction intensity refers to changes in mean expected duration (\(\beta^{0}\)). Selectivity then refers to how the covariates alter the expected duration of a taxon. This confusion has hopefully been cleared up with the updates to the manuscript.
  
  \textbf{I have made many comments on the pdf of the manuscript that will hopefully help the author clarify his writing. But in this letter, I would like to just summarize a couple of major issues that I noticed.}

  Comments from the pdf have been integrated into the manuscript where appropriate. Here, I will only be responding to the major points present in the review.
  
  \textbf{First, I think much of the details of the methods should not be hidden in the appendices, even if the appendices are part of the paper and not online supplements. I found it extremely difficult to know what the author has done because of the abbreviated manner in which the methods and materials were described. I made suggestions as to how to make the methods more transparent. I also strongly recommend that the STAN code be submitted with the ms as an online supplement. Code is often clearer than (or at least help to clarity) writing, even if the readers do not run the code. Along those lines as well, the epicontinental and open-ocean assignments should also be supplied where not previously published. }

  The Methods section has been completely over-hauled so that none of the methods are in an Appendix. The entire code base along with the dataset are now available as a Zenodo archive. 
  
  \textbf{The second larger issue is one of sampling bias. The author mentioned briefly that he assumes there is both left and right censorship, but it is unclear how biased sampling (that truncates genus durations and geographic ranges) is taken into account. Some of the conclusions drawn can be the effect of sampling bias rather than a “true” underlying pattern. For instance, it is possible (although I do not believe it is all there is to it) that a genus that has high sampling probability throughout, will also have a longer observed duration and a greater observed geographic range. Hence, observed genus durations will be positively correlated with observed geographic ranges or observed extinction risk negatively correlated with observed geographic ranges, no matter what statistical methods one might use. Michael Foote, Charles Marshall, Steve Wang and many others, including yours truly, have written about these issues in different contexts. It would be unfair to ask for a different piece of research (one could look at extinction risk while simultaneously modeling sampling, which I feel would be ideal for a study like this), but if you could use an existing approach (from Solow or Marshall or Wang.. …see his most recent papers which I do not insert below because I do no have internet here where I am working to finish this review) to estimate times of extinction to have an estimate of true duration, instead of using the data of genera in stages as given, that would be a very great improvement, and very doable?}

  Sampling is now discussed in the Introduction; see lines 143-167. % CHANGE

  The Solow and Marshall type methods are not ideal for analysis at this temporal resolution; if there were many more sampling horizons per unit time (smaller than stage) then they would be more appropriate. I chose to instead quantify the biasing effect of sampling on observed duration. How this is an alternative to these suggested methods or some form of Hidden Markov Model (i.e. CJS model) for studying survival is now discussed at length in the Methods and Discussion; see lines 236-251, 318-345, and 547-558. Importantly, sampling has a smaller magnitude of effect than the effect of geographic range and both of the effects of environmental preference; see lines 458-479. % CHANGE


  \textbf{The third is the justification/discussion of environmental preference and body size. The author used body size data from Payne et al. But for readers unfamiliar with that paper, it is not possible to know what the data actually are. The sources of the data and hence the underlying assumptions should be laid out for the ease of interpretation. The “lack” of correlation could be due to the nature of the measures rather than a true biological signal (but hard to judge not knowing the data). Also, the author discusses “survival of the unspecialized” using one environmental axis (which is more a sampled environment), and seems to discuss that as capturing the spectrum of “niche occupation”. In this context, I recommend this paper to read Houle, D., Pelabon, C., Wagner, G.P. \& Hansen, T.F. 2011 Measurement and meaning in biology. Quarterly Review of Biology 86, 3-34. It is on the meaning of the things we measure, mainly with quantitative genetics as a case study but infinitely applicable to interpretations in paleobiology. Please look at at least the list of ten “commandments” in the end of the review. }

  The source and nature of the body size data has now been included in the manuscript; see line 233-235. Also, there is now a discussion of environmental preference as only a single axis in the Discussion; see lines 516-530. % CHANGE

  
  \textbf{Minor comments: I am not sure why both Weibull and exponential are used when the Weibull will reduce to the exponential. In general, the labels of the graphs need to have units and be labeled with words in addition to symbols. In summary, I think this paper has a few cool things in it, but the author really needs to think about biased sampling and I think with re-analyses using preexisting methods to give better estimates of genus durations (rather than using face value data) will really improve inferences and our understanding of the uncertainty of the inferences. The writing also needs a fair bit of work such that an informed reader can actually repeat the analyses. I have not looked closely at the discussion since I cannot evaluate the results, because of the issues I have mentioned. }
  
  The exponential model has been excluded from this analysis, I now focus exclusively on a single model. Figures and captions have been updated to increase clarity. Please look at the Discussion now as I explore many of the issues raised in this review there.



\end{letter}
\bibliographystyle{plain}
\bibliography{newbib}
\end{document}

