\documentclass[12pt,letterpaper]{article}

\usepackage{amsmath, amsthm}
\usepackage{graphicx,hyperref}
\usepackage{microtype, parskip}
\usepackage[comma,sort&compress]{natbib}
\usepackage{lineno}
\usepackage{docmute}
\usepackage{subcaption, multirow, morefloats}
\usepackage{wrapfig}

\frenchspacing

\captionsetup[subfigure]{position = top, labelfont = bf, textfont = normalfont, singlelinecheck = off, justification = raggedright}

\begin{document}
\section{Introduction}

The most basic statement about the fossil record is that it is incomplete and that what has been preserved is a (biased) subset of the biodiversity that once existed. This fundamental incomplete incompleteness of the record, along with worker effort biased towards certain intervals, are the two major problems to accurately modeling the fossil record.


Sampling in paleontology has two meanings: geological and statistical. The first is the rate at which an organism is preserved as a fossil, and the second is the rate of observing a fossil occurrence. In this study I focus on the second definition, which I call the occurrence rate.



Some organismal groups are considered to have comparable fossil records meaning that the occurrence patterns can be considered transitive.

This assumes that all members within a group are considered to have identical occurrence rates.

There is known differences in occurrence rate within groups as some members may be much more commonly occurring than others either because of biological abundance, differences in worker effort, or preservational biases.


The Bayesian hierarchical modeling approach used here explicitly models the occurrence rate of a given genus or class in relation to the entire record of fossil occurrences for the entire Phanerozoic. 




bayesian hierarchical modeling approach here is new for paleontology



\uppercase{notes and papers}

previous approaches to overcoming

Alroy2010c % fair sampling

a foote miller paper using rarefaction along with a million others

Jablonski1991 % using other records as proxy

Marshall % CIs on durations

Sadler1981

Wagner2007

Wang2004

Wang2012b

previous approaches to modeling

Alroy2014

Foote1996d

Foote1996e

Foote1997c

Foote1999a

Foote2001

Solow1997

Strauss1989

Wagner2013a


other

Foote2007a

Jernvall2002

Liow2007d

Lloyd2011

Lloyd2012b

Lloyd2012c

Lloyd2013

McGowan/Smith and McGowan

Sepkoski1975

Simpson2009

Wagner2000h


\end{document}
