\documentclass[12pt,letterpaper]{article}

\usepackage{amsmath, amsthm}
\usepackage{microtype, parskip}
\usepackage[comma,numbers,sort&compress]{natbib}
\usepackage{lineno}
\usepackage{docmute}
\usepackage{caption, subcaption, multirow, morefloats, rotating}
\usepackage{wrapfig}

\frenchspacing

\captionsetup[subfigure]{position = top, labelfont = bf, textfont = normalfont, singlelinecheck = off, justification = raggedright}

\begin{document}

\begin{abstract}
  How does ecology affect taxon duration? While the effect of geographic range on extinction risk is well documented, the effects of other traits are much less well documented. Here, I analyze patterns of Paleozoic brachiopod generic durations and how various biological traits are related to systematic differences in expected extinction risk. I analyze geographic range, affinity for epicontinental seas versus open ocean environments, and body size. Additionally, I allow for environmental affinity to have a nonlinear effect on duration. I do this in a hierarchical Bayesian modeling context, allowing me to directly model the possible interaction between the effects of biological traits and time of origination. I find that as extinction risk increases, the expected strength of the selection gradient on biological traits (except body size) increases. This manifests as greater expected differences in extinction risk for each unit change in geographic range and environmental preference during periods of high extinction risk, while a much flatter expected selection gradient during periods of low extinction risk. I find evidence for a nonlinear relationship between environmental preference and extinction risk such that intermediate affinities (``generalists'') have a lower expected extinction risk than either end members (``specialists''). Interestingly, as extinction risk increases, the peakedness of this relationship is expected to increases. These results demonstrate the importance of directly modeling the structure inherent in the observed data as a means to better understand which processes may have been driving the observed patterns of diversification.
\end{abstract}

\end{document}
