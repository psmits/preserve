\documentclass[12pt,letterpaper]{article}

\usepackage{amsmath, amsthm}
\usepackage{microtype, parskip}
\usepackage[comma,numbers,sort&compress]{natbib}
\usepackage{lineno}
\usepackage{docmute}
\usepackage{caption, subcaption, multirow, morefloats, rotating}
\usepackage{wrapfig}

\frenchspacing

\captionsetup[subfigure]{position = top, labelfont = bf, textfont = normalfont, singlelinecheck = off, justification = raggedright}

\begin{document}

\begin{abstract}
  How do difference in ecology affect differences in taxon durations? While the effect of geographic range on extinction risk is well document, the effects of other traits are much less well documented. Here, I analyze patterns of Paleozoic brachiopod generic durations and how various taxon traits are related to systematic differences in expected extinction risk. I analyze geogrpahic range, affinity for epicontitnental seas versus open ocean environments, and body size. Additionally, I allow for environmental affinity to have a possibly nonlinear effect on duration. I do this in a hierarchical Bayesian modeling context, allowing me to directly model the possible correlation between the effects of these traits between origination cohorts. I find evidence for a nonlinear relationship between environmental preference and extinction risk such that intermediate affinities (``generalists'') have a lower expected extinction risk than either end members (``specialists''). I also find support for correlation between baseline extinction risk and the effect of geographic range, such that within cohorts with high baseline extinction risk the effect of geographic range is larger than in cohorts with low baseline extinction risk. The other taxon traits, however, do not demonstrate a similarly strong correlation. This results is contrary to previously proposed macroevolutionary mechanisms regarding changes in the effects of taxon traits on survival. These results demonstrate the importance of directly modeling the structure inherent in the observed data as a means to better understand what possible processes may have been driving the observed patterns of diversification.
\end{abstract}

\end{document}
