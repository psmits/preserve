\documentclass[12pt,letterpaper]{article}

\usepackage{amsmath, amsthm}
\usepackage{microtype, parskip}
\usepackage[comma,numbers,sort&compress]{natbib}
\usepackage{lineno}
\usepackage{docmute}
\usepackage{caption, subcaption, multirow, morefloats, rotating}
\usepackage{wrapfig}

\frenchspacing

\captionsetup[subfigure]{position = top, labelfont = bf, textfont = normalfont, singlelinecheck = off, justification = raggedright}

\begin{document}

\begin{abstract}
  While the effect of geographic range on extinction risk is well documented, the effects of other traits are less well known. Here, I analyze patterns of Paleozoic brachiopod genus durations and their relationship to geographic range, affinity for epicontinental seas versus open ocean environments, and body size. Additionally, I allow for environmental affinity to have a nonlinear effect on duration. Using a hierarchical Bayesian modeling approach, I also model the possible interaction between the effects of the biological traits and a taxon's time of origination. I find evidence that as extinction risk increases, the expected strength of the selection gradient on biological traits (except body size) increases. This manifests as greater expected differences in extinction risk for each unit change in geographic range and environmental preference during periods of high extinction risk, as opposed to a much flatter expected selection gradient during periods of low extinction risk. I find weak evidence for a nonlinear relationship between environmental preference and extinction risk such that ``generalists'' have a lower expected extinction risk than either ``specialists''. Interestingly, I find that as extinction risk increases, the peakedness of this relationship is expected to increases as well. These results demonstrate the importance of directly modeling the structure inherent in the observed data as a means to better understand which processes may have been driving the observed patterns of diversification.
\end{abstract}

\end{document}
