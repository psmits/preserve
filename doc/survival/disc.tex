\documentclass[12pt,letterpaper]{article}

\usepackage{amsmath, amsthm}
\usepackage{microtype, parskip}
\usepackage[comma,numbers,sort&compress]{natbib}
\usepackage{lineno}
\usepackage{docmute}
\usepackage{caption, subcaption, multirow, morefloats, rotating}
\usepackage{wrapfig}

\frenchspacing

\captionsetup[subfigure]{position = top, labelfont = bf, textfont = normalfont, singlelinecheck = off, justification = raggedright}

\begin{document}
\section{Discussion}

% hypotheses
%   Jablonski1986 hypothesis
My results demonstrate that both the effects of geographic range and the peakedness/concavity of environmental preference are both negatively correlated with baseline extinction risk, meaning that as baseline extinction risk increases the effect sizes of both these traits are expected to increase (Fig. \ref{fig:corr}). This result supports neither of the two proposed macroevolutionary mechanisms for how the effects of biological traits should correlate with extinction risk. The observed correlation between the effects as well as between the effects and baseline extinction risk instead implies that as baseline extinction risk increases, the strength of the selection gradient on biological traits (except body size) increases. This manifests as greater differences in extinction risk for each unit difference in the biological covariates during periods of high extinction risk, while a much flater selection gradient during periods of low extinction risk.

%   Miller2009a results
There are two mass extinction events that are captured within the time frame considered here: the Ordovician-Silurian and the Frasnaian-Famennian. The cohorts bracketing these events are worth considering in more detail.

% write about the mass extinctions \ref{fig:env_cohort}

The proposed mechanism for the end Ordovician mass extinction is a decrease in sea level and the draining of epicontinental seas due to protracted glaciation \citep{Sheehan2001b,Johnson1974}. My results are broadly consistent with this scenario with both epicontinental and open-ocean specialists having a much lower expected duration than intermediate taxa (Fig. \ref{fig:env_cohort}). All of the stages between the Darriwillian and the Llandovery, except the Hirnantian, have a greater than expected probability that \(f(v)\) is concave down. The pattern for the Darriwilian, which marks the supposed start of Ordovician glacial activity, demonstrates that taxa tend to favor open-ocean environments are expected to have a greater duration than either intermediate of epicontinental specialists, in decreasing order.

For nearly the entire Devonian estimates of \(f(v)\) indicate that one of the environmental end members is favored over the other end member of intermediate preference (Fig. \ref{fig:env_cohort}). For the Givetian though the end of the Devonian and into the Tournasisian, I find that epicontinental favoring taxa are expected to have a greater duration than either intermediate or open-ocean specialists. Additionally, for nearly the entire Devonian except the Eifelian and through the Visean, the cohort-specific estimates of \(f(v)\) are concave-up. This is the opposite pattern than what is expected (Fig. \ref{fig:env_mean}. This result, however, seems to reflect the intensity of the nearly-linear difference in expected duration across the range of \(v)\) as opposed to an inversion of the expected curvilinear pattern.

There is an approximate 86\% posterior probability that taxa with intermediate environmental preferences have a lesser extinction risk than either end members, the over all curvature of \(f(v_{i})\) is not very peaked meaning that this relationship does not lead to very strong differences in extinction risk (Fig. \ref{fig:env_mean}). This result supports the hypothesis that, in general, environmental generalists survive for longer than environmental specialists \citep{Simpson1944,Liow2004a,Liow2007b,Nurnberg2013a,Nurnberg2015}.

The variance in estimate of the overall \(f(v_{i})\) reflects the large between cohort variance in cohort specific estimates of \(f(v_{i})\) (Fig. \ref{fig:env_cohort}). Given that there is only a 86\% posterior probability that the expected overall estimate of \(f(v_{i})\), it is not surprising that there are some stages where the theorized relationship is in fact reversed. Additionally, as discussed earlier, there are some stages where \(f(v_{i})\) does not resemble the theorized nonlinear relation with the optimum in the middle but instead is highly skewed or effectively linear (Fig. \ref{fig:env_cohort}). 

These results do not necessarily refute ``survival of the unspecialized'' as a time-invariant generalization, but instead demonstrate how, while the expected group-level estimate of \(f(v_{i})\) might favor one hypothesis, there is still enough variability between cohorts so that in some realizations this pattern may not hold or can even be reversed. These results are also consistent with aspects of \citet{Miller2009a} who found that the effect of environmental preference on extinction risk was quite variable and without obvious patterning during times of background extinction.

% defense
%   species:genus?
%   difficulty towards tails, but that's to be expected
%     this model is about expectations, not tails/extreme events
%     this model is ok for the main part of the data
%     though, of course, this model has a long way to go (all models are false)

% future direction
This model can be improved through either increasing the number of analyzed taxon traits, expanding the hierarchical structure of the model to include other major taxonomic groups of interest, and the inclusion of explicit phylogenetic relationships between the taxa in the model as an additional hierarchical effect.

%   other measures of ecology? affixing strategy a la Alexander1977
An example taxon trait that may be of particular interest is the affixing strategy or method of interaction with the substrate of the taxon. This trait has been found to be related to brachiopod survival \citep{Alexander1977} so its inclusion may be of particular interest.

%   comparison with other major groups in hierarchical model
It is theoretically possible to expand this model to allow for comparisons within and between major taxonomic groups. This approach would better constrain the brachiopod estimates while also allowing for estimation of similarities and differences in cross-taxonomic patterns. The major issue surrounding this particular expansion involves finding an similarly well sampled taxonomic group that is present during the Paleozoic. Example groups include Crinoidea, Ostracoda, and other ``Paleozoic'' groups \citep{SepkoskiJr.1981a}.

%   integration of phylogenetic information/taxonomic component
%     taxon traits are more than likely heritable
%     what aspect of variation is explained just by 
%     see Smits Submitted
Taxon traits like environmental preference or geographic range \citep{Jablonski1987,Hunt2005b} are most likely heritable, at least phylogenetically \citep{Lynch1991,Housworth2004}. Without phylogenetic context, this analysis assumes that differences in extinction risk between taxa are independent of those taxa's shared evolutionary history \citep{Felsenstein1985b}. In contrast, the origination cohorts only capture shared temporal context. The inclusion of phylogenetic context as an addition individual level hierarchical structure independent of origination cohort would allow for determining how much of the observed variability is due to shared evolutionary history versus actual differences associated with these taxonomic traits. For example, it has been shown that phylogeny contribute non-trivially to differences in mammal species durations \uppercase{Smits in prep}.

% concluding statements
In summary, patterns of Paleozoic brachiopod survival were analyzed using a fully Bayesian hierarchical survival modelling approach. Using a varying-slopes, varying-intercepts approach I am able to model both the overall mean effect of biological covariates on extinction risk while also modeling the correlation between origination cohort-specific estimates of covariate effects. I find that as baseline extinction risk increases, the strength of the selection gradient on biological traits (except body size) increases. This manifests as greater differences in extinction risk for each unit difference in the biological covariates during periods of high extinction risk, while a much flatter selection gradient during periods of low extinction risk. I also find some support for ``survival of the unspecialized'' \citep{Simpson1944,Liow2004a,Liow2007b,Nurnberg2013a,Nurnberg2015} as a general characterization of the effect of environmental preference on extinction risk (Fig. \ref{fig:env_mean}), though there is heterogeneity between origination cohorts (Fig. \ref{fig:env_cohort}). Generally, this study demonstrates the advantages of a hierarchical Bayesian framework for taking into account the structured nature of the data. Future studies of structured data should adopt similar strategies in order to best model our knowledge instead of ignoring that structure which can lead to poor and/or incorrect inference.

\end{document}
