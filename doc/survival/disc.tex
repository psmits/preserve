\documentclass[12pt,letterpaper]{article}

\usepackage{amsmath, amsthm}
\usepackage{microtype, parskip}
\usepackage[comma,numbers,sort&compress]{natbib}
\usepackage{lineno}
\usepackage{docmute}
\usepackage{caption, subcaption, multirow, morefloats, rotating}
\usepackage{wrapfig}

\frenchspacing

\captionsetup[subfigure]{position = top, labelfont = bf, textfont = normalfont, singlelinecheck = off, justification = raggedright}

\begin{document}
\section{Discussion}

% hypotheses
%   Jablonski1986 hypothesis
The results presented here are in an interesting combination of congruence and contrast with those of \citet{Jablonski1986}.  These results are partially consistent with \citet{Jablonski1986} because during periods of higher baseline extinction risk the effect of geographic range is greater than the effects of other taxon traits, though the macroevolutionary mechanism proposed by \citet{Jablonski1986} is not supported by my results. \citet{Jablonski1986} hypothesized that as extinction risk increases, the effects of all taxon traits except for geographic range size decrease in size. My results instead demonstrate that the only effect that has strong correlation with baseline extinction risk is that of geographic range (Fig. \ref{fig:corr}). Because of this, instead of the effects of other taxon traits decreasing with increasing baseline extinction risk, the effect of geographic range increases while the effects of other taxon traits have no strong correlation with baseline extinction risk. Keep in mind, however, that this is a model of continuous rates and not a model of the differences between background and mass extinction.

%   Miller2009a results
There are two mass extinction events that are captured within the time frame considered here: the end Ordovician, and the end Devonian. The nature of the analysis used here means that the effect of the mass extinction is only captured with any clarity by the survival patterns of the origination cohort that immediately proceeds the event. These two cohorts are Hirnantian and Frasnian for the end Ordovician and end Devonian, respectively.

For the Hirnantian, there is a 57\% posterior probability that \(f(v_{i})\) is nonlinear and  that taxa which tend to favor epicontinental seas are expected to survive longer than those favoring open ocean environments. Similarly, for the Frasnian there is a 47\% posterior probability that \(f(v_{i})\) is nonlinear and taxa which tend to favor epicontinental seas are expected to survive longer than those favoring open ocean environments. These probabilities were calculated as the percent of posterior draws of \(f(v_{i})\) which have their optima greater than 0, or towards favoring epicontinental. 

In the Frasnian, however, there is evidence that \(f(v_{i})\) is effectively linear and taxa which favor epicontinental seas will be longer lived than those which favor open ocean environements (18\% prosterior probability). This probability was calculated as the probability that the inflection point/optima of \(f(v_{i}\) will be between -1 and 1 which is a genera desciription of if there is any non-linearity around values corresponding ``generalists.''

In the stages following the mass extinction, however, this story is slightly different. The Rhuddanian, which just follows the end Ordovician extinction, there is a noticible non-linear relationship, where taxa which favor open open environments tend to have a greater duration multiplier than those which favor epicontinental seas. This is consistent with the proposed mechanism of the end Ordovician mass extinction where the sea level dropped and the epicontinental seas were drained \citep{Johnson1974,Sheehan2001b}. 

For the Famennian, which follows the end Devonian mass extinction, there is an almost linear relationship where taxa which favor epicontinental seas have a much greater duration multiplier than any other environmental affinity. This means that the observation of \citet{Miller2009a} is a consistent descriptor of the long term extinction risk of taxa originating in stages just following a mass extinction.

These results indicate weak evidence for the observation of \citet{Miller2009a} that epicontinental seas are favored at mass extinction boundaries they analyzed. For stages preceding mass extinction, both affinity end members are approximately equal in extinction risk. For stages following mass extinctions, however, the results observed here for the Famennian are consistent with those of \citet{Miller2009a}.

This analysis, however, may not be capturing the patterns analyzed in \citet{Miller2009a} as my focus is on long term taxonomic duration between and within origination cohorts and not patterns of instantaneous extinction rates between stages. Additionally, my censoring criteria are specifically for smoothing out this kind of disruption, regardless of cohort. This censoring pattern may be the reason for the difference in relationship of the stage preceding a mass extinction and the stage following it. Also, as said before, this is not a model of background versus mass extinction, but instead it is a model of differences in long term extinction risk. Given of these caveats, it is hard to make strong conclusions directly about the results of \citet{Miller2009a}.

There is an approximate 75\% posterior probability that taxa with intermediate environmental preferences have a lesser extinction risk than either end members, the over all curvature of \(f(v_{i})\) is not very peaked meaning that this relationship does not lead to very strong differences in extinction risk (Fig. \ref{fig:env_mean}). This result supports the hypothesis that, in general, environmental generalists survive greater than environmental specialists \citep{Simpson1944,Liow2004a,Liow2007b,Nurnberg2013a,Nurnberg2015}.

The variance in estimate of the overall \(f(v_{i})\) reflects the large between cohort variance in cohort specific estimates of \(f(v_{i})\) (Fig. \ref{fig:env_cohort}). Given that there is only a 75\% posterior probability that the expected overall estimate of \(f(v_{i})\), it is not surprising that there are some stages where the theorized relationship is in fact reversed. Additionally, as discussed earlier, there are some stages where \(f(v_{i})\) does not resemble the theorized nonlinear relation with the optimum in the middle but instead is highly skewed or effectively linear (Fig. \ref{fig:env_cohort}). 

These results do not necessarily refute ``survival of the unspecialized'' as a time-invariant generalization, but instead demonstrate how, while the expected group-level estimate of \(f(v_{i})\) might favor one hypothesis, there is still enough variability between cohorts so that in some realizations this pattern may not hold or can even be reversed. These results are also consistent with aspects of \citet{Miller2009a} who found that the effect of environmental preference on extinction risk was quite variable and without obvious patterning during times of background extinction.

% defense
%   species:genus?
%   difficulty towards tails, but that's to be expected
%     this model is about expectations, not tails/extreme events
%     this model is ok for the main part of the data
%     though, of course, this model has a long way to go (all models are false)

% future direction
This model can be improved through either increasing the number of analyzed taxon traits, expanding the hierarchical structure of the model to include other major taxonomic groups of interest, and the inclusion of explicit phylogenetic relationships between the taxa in the model as an additional hierarchical effect.

%   other measures of ecology? affixing strategy a la Alexander1977
An example taxon trait that may be of particular interest is the affixing strategy or method of interaction with the substrate of the taxon. This trait has been found to be related to brachiopod survival \citep{Alexander1977} so its inclusion may be of particular interest.

%   comparison with other major groups in hierarchical model
It is theoretically possible to expand this model to allow for comparisons within and between major taxonomic groups. This approach would better constrain the brachiopod estimates while also allowing for estimation of similarities and differences in cross-taxonomic patterns. The major issue surrounding this particular expansion involves finding an similarly well sampled taxonomic group that is present during the Paleozoic. Example groups include Crinoidea, Ostracoda, and other ``Paleozoic'' groups \citep{SepkoskiJr.1981a}.

%   integration of phylogenetic information/taxonomic component
%     taxon traits are more than likely heritable
%     what aspect of variation is explained just by 
%     see Smits Submitted
Taxon traits like environmental preference or geographic range \citep{Jablonski1987,Hunt2005b} are most likely heritable, at least phylogenetically \citep{Lynch1991,Housworth2004}. Without phylogenetic context, this analysis assumes that differences in extinction risk between taxa are independent of those taxa's shared evolutionary history \citep{Felsenstein1985b}. In contrast, the origination cohorts only capture shared temporal context. The inclusion of phylogenetic context as an addition individual level hierarchical structure independent of origination cohort would allow for determining how much of the observed variability is due to shared evolutionary history versus actual differences associated with these taxonomic traits. It has been shown that phylogeny contribute non-trivially to differences in mammal species durations \uppercase{Smits in prep}.

% concluding statements
%   different mechanism than proposed by Jablonski1986, but same overall observation
%   no conclusive support for either Miller2009a or its converse
%   support for survival of the unspecialized
In summary, patterns of Paleozoic brachiopod survival were analyzed using a fully Bayesian hierarchical survival modelling approach. Using a varying-slopes, varying-intercepts approach I am able to model both the overall mean effect of biological covariates on extinction risk while also modeling the correlation between origination cohort-specific estimates of covariate effects. I find that cohort baseline extinction risk is correlated with the cohort-specific effect of geographic range on extinction risk. Specifically, as baseline extinction risk increases, the strength of the effect of geographic range increases. Other taxon traits, however, were not strongly correlated with baseline extinction risk. This result is not consistent with the proposed macroevoloutionary mechanism for the observed pattern that geographic range is the most important taxon trait during periods of high extinction risk \citet{Jablonski1986}. I also find some support for ``survival of the unspecialized'' \citep{Simpson1944,Liow2004a,Liow2007b,Nurnberg2013a,Nurnberg2015} as a general characterization of the effect of environmental preference on extinction risk (Fig. \ref{fig:env_mean}), though there is heterogeneity between origination cohorts (Fig. \ref{fig:env_cohort}). Generally, this study demonstrates the advantages of a hierarchical Bayesian framework for taking into account the structured nature of the data. Future studies of structured data should adopt similar strategies in order to best model our knowledge instead of ignoring that structure which can lead to poor and/or incorrect inference.

\end{document}
