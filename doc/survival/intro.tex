\documentclass[12pt,letterpaper]{article}

\usepackage{amsmath, amsthm}
\usepackage{microtype, parskip}
\usepackage[comma,numbers,sort&compress]{natbib}
\usepackage{lineno}
\usepackage{docmute}
\usepackage{caption, subcaption, multirow, morefloats, rotating}
\usepackage{wrapfig}

\frenchspacing

\captionsetup[subfigure]{position = top, labelfont = bf, textfont = normalfont, singlelinecheck = off, justification = raggedright}

\begin{document}
\section{Introduction}

How do differences in taxon traits affect differences in extinction risk? \citet{Jablonski1986} hypothesizes that as baseline extinction risk increases, the effects of taxon traits on survival should decrease. The trait where this pattern should be weakest or absent is geographic range \citep{Jablonski1986}. Taxon traits are defined here as descriptors of a taxon's adaptive zone which is the set of biotic--biotic and biotic--abiotic interactions that a taxon can and does experience. In effect, these are descriptors of a taxon's broad-sense ecology.

\citet{Jablonski1986} phrased this hypothesis in terms of background versus mass extinction, but it is readily transferable to a continuous variation framework as there is no obvious distinction in terms of extinction rate between these two states \citep{Wang2003}. I adopt a continuous variation framework as this is more amenable for modeling the relationship between taxon traits and extinction risk. Additionally, the \citet{Jablonski1986} hypothesis has strong model structure requirements in order to test it's proposed macroevolutionary mechanism. Not only do the taxon trait effects need to be modeled but the relationships between these effects need to be modeled as well. This level of complexity is possible in a varying intercepts, varying slopes hierarchical model as discussed below.

I choose to model differences in taxon durations because trait based differences in extinction risk should manifest as differences in taxon durations. For example, species with a beneficial trait should survive longer, on average, than a species without that beneficial trait. Conceptually, taxon survival can be considered an aspect of ``taxon fitness'' along with lineage specific branching/origination rate \citep{Cooper1984,Palmer2012}. The analysis of taxon durations, time from origination to extinction, falls under the purview of survival analysis, a field of applied statistics commonly used in health care \citep{Klein2003} but has a long history in paleontology \citep{Simpson1944,Simpson1953,VanValen1973,VanValen1979}.

Geographic range is widely considered the most important taxon trait for estimating differences in extinction risk at nearly all times \citep{Jablonski1986,Jablonski1987,Jablonski2003,Payne2007}. This is logical because if mortality is randomly distributed spatially, a taxon with a large geographic range is less likely to be completely wiped out than a taxon with restricted range. This is a strong prediction based on strong prior evidence that should be incorporated into any model of extinction risk in order to best represent our knowledge.

\citet{Miller2009a} demonstrated that during mass extinctions taxa associated with open ocean environments tend to have a greater extinction risk than those taxa associated with epicontinental seas. During periods of background extinction, however, they found no consistent difference between taxa favoring either environment. Because of this study, the following prediction for survival patterns can be made: as extinction risk increases, taxa associated with open ocean environments should generally increase in extinction risk versus those that favor epicontinental seas.

There is also a possible nonlinear relationship between environmental preference and taxon duration. A long standing hypothesis is that generalists or unspecialized taxa will have greater survival than specialists \citep{Simpson1944,Liow2004a,Liow2007b,Nurnberg2013a,Nurnberg2015,Baumiller1993} \uppercase{Smits, in prep}. A simple expectation for continuous traits is that this nonlinearity would manifest as a concave down function, with taxa with intermediate trait values having a greater expected durations than taxa with high or low trait values. By utilizing a continuous measure of environmental preference, it is relatively straight forward to allow for a nonlinear effect in a model of taxon durations/extinction risk.

I adopt a hierarchical Bayesian survival modelling approach for multiple reasons, one being that it represents a conceptual and statistical unification of the paleontological dynamic and cohort survival analytic approaches \citep{VanValen1973,VanValen1979,Raup1978,Raup1975,Foote1988,Baumiller1993,Simpson2006}. By using a Bayesian framework I am best able to quantify the uncertainty inherent in the estimates of the effects of taxon traits on survival, especially in cases where the covariates of interest (taxon traits) are themselves known with error. 

Hierarchical modelling, sometimes called ``mixed-effects modeling,'' is a statistical approach which explicitly takes into account the structure of the observed data in order to model both the with and between group variance \citep{Gelman2013d,Gelman2007}. In this case, origination cohorts are the groups and the mean survival model corresponds to the dynamic survivorship model. This leads to simultaneous dynamic and cohort style analysis. The units of study (e.g. genera) each belong to a single grouping (e.g. origination cohort) that are also considered draws from a shared probability distribution (e.g. all cohorts, observed and unobserved). The group level parameters are then estimated simultaneously as the other parameters of interest (e.g. covariate effects) \citep{Gelman2013d}. The subsequent estimates are partially pooled together, where parameters from groups with large samples or effects remain large while those of groups with small samples or effects are pulled towards the overall group mean. 

This partial pooling is one of the greatest advantages of hierarchical modeling. By letting the groups ``support'' each other, parameter estimates then better reflect our actual uncertainty. Additionally, this partial pooling helps control for multiple comparisons and possibly spurious results as effects with little support are drawn towards the overall group mean \citep{Gelman2013d,Gelman2007}. 

All covariate effects (regression coefficients), as well as the intercept term (baseline extinction risk), were allowed to vary by group (origination cohort). The covariance/correlation between covariate effects was also modeled. This hierarchical structure allows inference for how covariates effects may change with respect to each other while simultaneously estimating the effects themselves, correctly propagating our uncertainty. Additionally, instead of relying on potentially biased point estimates of environmental affinity, I adopt a measurement error modeling approach where environmental affinity is a continuous measure of the difference in the taxon's probabilistic environmental occurrence pattern and the background probabilistic environmental occurrence pattern. This approach correctly propagates our uncertainty, which leads to more valid posterior inference. 

\end{document}
