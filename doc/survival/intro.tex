\documentclass[12pt,letterpaper]{article}

\usepackage{amsmath, amsthm}
\usepackage{microtype, parskip}
\usepackage[comma,numbers,sort&compress]{natbib}
\usepackage{lineno}
\usepackage{docmute}
\usepackage{caption, subcaption, multirow, morefloats, rotating}
\usepackage{wrapfig}

\frenchspacing

\captionsetup[subfigure]{position = top, labelfont = bf, textfont = normalfont, singlelinecheck = off, justification = raggedright}

\begin{document}
\section{Introduction}

How do biological traits affect extinction risk? \citet{Jablonski1986} observed that during periods of high extinction risk, the effects of biological traits on survival decreased in size. However, this pattern was weakest/absent in the effect of geographic range on survival \citep{Jablonski1986}. Biological traits are defined here as descriptors of a taxon's adaptive zone, which is the set of biotic--biotic and biotic--abiotic interactions that a taxon can experience \citep{Simpson1944}. In effect, these are descriptors of a taxon's broad-sense ecology.

\citet{Jablonski1986} phrased his conclusions in terms of background versus mass extinction, but this scenario is readily transferable to a continuous variation framework as there is no obvious distinction in terms of extinction rate between these two states \citep{Wang2003}. Additionally, the \citet{Jablonski1986} scenario has strong model structure requirements in order to test its proposed macroevolutionary mechanism; not only do the taxon trait effects need to be modeled, but the correlation between the trait effects need to be modeled as well. 

There are two end-member macroevolutionary mechanisms which may underlie the pattern observed by \citet{Jablonski1986}: the effect of geographic range on predictive survival remains constant and those of other biological traits decrease, and the effect of geographic range in predicting survival increases and those of other biological traits stay constant. Reality, of course, may fall somewhere along the continuum between these two opposites.

I model brachiopod taxon durations because trait based differences in extinction risk should manifest as differences in taxon durations. Namely, a taxon with a beneficial trait should survive longer, on average, than a taxon without that beneficial trait. Conceptually, taxon survival can be considered an aspect of ``taxon fitness'' along with expected lineage specific branching/origination rate \citep{Cooper1984,Palmer2012}. Brachiopods are an ideal group for this study as they are are well known for having an exceptionally complete fossil records \citep{Foote2000a}. Specifically, I focus on the brachiopod record from most of the Paleozoic, specifically from the start of the Ordovician (approximately 485 Mya) through the end Permian (approximately 252 Mya) as this represents the time of greatest global brachiopod diversity \citep{Alroy2010}.

he analysis of taxon durations, or time from origination to extinction, falls under the purview of survival analysis, a field of applied statistics commonly used in health care \citep{Klein2003} but has a long history in paleontology \citep{Simpson1944,Simpson1953,VanValen1973,VanValen1979}.

Geographic range is widely considered the most important taxon trait for estimating differences in extinction risk at nearly all times with large geographic range associated with low extinction risk \citep{Jablonski1986,Jablonski1987,Jablonski2003,Payne2007}. I expect this to hold true nearly always.

\citet{Miller2009a} demonstrated that during several mass extinctions taxa associated with open-ocean environments tend to have a greater extinction risk than those taxa associated with epicontinental seas. During periods of background extinction, however, they found no consistent difference between taxa favoring either environment. These two environment types represent the primary environmental dichotomy observed in ancient marine systems \citep{Miller2009a,Peters2008,Sheehan2001b}. 

Epicontinental seas are a shallow-marine environment where the ocean has spread over the surface of a continental shelf with a depth typically less than 100m. In contrast, open-ocean coastline environments have much greater variance in depth, do not cover the continental shelf, and can persist during periods of low sea level. Because of this, it is strongly expected that taxa which favor epicontinental seas would be at great risk during periods of low sea levels, such as during glacial periods, where these seas are drained. During the Paleozoic (approximately 541--252 My), epicontinental seas were widely spread globally but declined over the Mesozoic (approximately 252--66 My) and eventually diminished disappearing during the Cenozoic (approximately 66--0 My) as open-ocean coastlines became the dominant shallow-marine setting \citep{Peters2008,Miller2009a,Johnson1974}. 

Given the above information, I predict that as extinction risk increases, taxa associated with open-ocean environments should generally increase in extinction risk versus those that favor epicontinental seas. Additionally, there is a possible nonlinear relationship between environmental preference and taxon duration. A long standing hypothesis is that generalists or unspecialized taxa will have greater survival than specialists \citep{Simpson1944,Liow2004a,Liow2007b,Nurnberg2013a,Nurnberg2015,Baumiller1993}. In this analysis I allowed for environmental preference to possibly have a parabolic effect on taxon duration 

Body size, measured as shell length \citep{Payne2014}, was also considered as a potentially informative covariate. Body size is a proxy for metabolic activity and other correlated life history traits \citep{Payne2014}. There is no strong hypothesis of how body size effects extinction risk in brachiopods, meaning a positive, negative, or zero effect are all plausible. 

I adopt a hierarchical Bayesian survival modeling approach, which represents a conceptual and statistical unification of the paleontological dynamic and cohort survival analytic approaches \citep{VanValen1973,VanValen1979,Raup1978,Raup1975,Foote1988,Baumiller1993,Simpson2006}. By using a Bayesian framework I am able to quantify the uncertainty inherent in the estimates of the effects of biological traits on survival, especially in cases where the covariates of interest (i.e. biological traits) are themselves known with error. 

\end{document}
