\documentclass[12pt,letterpaper]{article}

\usepackage{amsmath, amsthm}
\usepackage{microtype, parskip}
\usepackage[comma,numbers,sort&compress]{natbib}
\usepackage{lineno}
\usepackage{docmute}
\usepackage{caption, subcaption, multirow, morefloats, rotating}
\usepackage{wrapfig}

\frenchspacing

\captionsetup[subfigure]{position = top, labelfont = bf, textfont = normalfont, singlelinecheck = off, justification = raggedright}

\begin{document}
\section{Introduction}

How do taxon traits affect extinction risk? \citet{Jablonski1986} hypothesizes that as baseline extinction risk increases, the effects of taxon traits on survival should decrease. The trait where this pattern should be weakest or absent is geographic range \citep{Jablonski1986}. Taxon traits are defined here as descriptors of a taxon's adaptive zone, which is the set of biotic--biotic and biotic--abiotic interactions that a taxon can experience. In effect, these are descriptors of a taxon's broad-sense ecology.

\citet{Jablonski1986} phrased this hypothesis in terms of background versus mass extinction, but it is readily transferable to a continuous variation framework as there is no obvious distinction in terms of extinction rate between these two states \citep{Wang2003}. I adopt a continuous variation framework as this is more amenable for modeling the relationship between taxon traits and extinction risk. Additionally, the \citet{Jablonski1986} hypothesis has strong model structure requirements in order to test its proposed macroevolutionary mechanism. Not only do the taxon trait effects need to be modeled, but the relationships between these effects need to be modeled as well. 

I model in taxon durations because trait based differences in extinction risk should manifest as differences in taxon durations. For example, species with a beneficial trait should survive longer, on average, than a species without that beneficial trait. Conceptually, taxon survival can be considered an aspect of ``taxon fitness'' along with lineage specific branching/origination rate \citep{Cooper1984,Palmer2012}. The analysis of taxon durations, time from origination to extinction, falls under the purview of survival analysis, a field of applied statistics commonly used in health care \citep{Klein2003} but has a long history in paleontology \citep{Simpson1944,Simpson1953,VanValen1973,VanValen1979}.

Geographic range is widely considered the most important taxon trait for estimating differences in extinction risk at nearly all times \citep{Jablonski1986,Jablonski1987,Jablonski2003,Payne2007}. 

\citet{Miller2009a} demonstrated that during several mass extinctions taxa associated with open ocean environments tend to have a greater extinction risk than those taxa associated with epicontinental seas. During periods of background extinction, however, they found no consistent difference between taxa favoring either environment. Because of this study, the following prediction for survival patterns can be made: as extinction risk increases, taxa associated with open ocean environments should generally increase in extinction risk versus those that favor epicontinental seas.

There is also a possible nonlinear relationship between environmental preference and taxon duration. A long standing hypothesis is that generalists or unspecialized taxa will have greater survival than specialists \citep{Simpson1944,Liow2004a,Liow2007b,Nurnberg2013a,Nurnberg2015,Baumiller1993} \uppercase{Smits, in prep}. 
%By utilizing a continuous measure of environmental preference, it is relatively straight forward to allow for a nonlinear effect in a model of taxon durations/extinction risk.

I adopt a hierarchical Bayesian survival modeling approach, which represents a conceptual and statistical unification of the paleontological dynamic and cohort survival analytic approaches \citep{VanValen1973,VanValen1979,Raup1978,Raup1975,Foote1988,Baumiller1993,Simpson2006}. By using a Bayesian framework I am able to quantify the uncertainty inherent in the estimates of the effects of taxon traits on survival, especially in cases where the covariates of interest (taxon traits) are themselves known with error. 
\end{document}
