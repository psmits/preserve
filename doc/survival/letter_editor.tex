\documentclass{letter}
\usepackage{microtype}
%\usepackage{letterbib}
%\usepackage{natbib}
\frenchspacing

\signature{Peter D Smits}
\address{Committee on Evolutionary Biology \\ University of Chicago \\
1025 E. 57th Street \\ Culver Hall 402 \\ Chicago, IL 60637 \\
psmits@uchicago.edu}

\begin{document}
\begin{letter}{Editor \\ \textit{Evolution}}
  \opening{Dear Editor,}

  Please find enclosed the manuscript entitled ``The interplay between extinction intensity and selectivity: correlation in trait effects on taxonomic survival.'' In this study I analyzed the effect of biological traits on brachiopod genus durations in order to understand how increases or decreases in extinction intensity influence the selectivity, or importance, of these traits. The biological traits analyzed included body size, geographic range size, and environmental preference.
  
  Standing wisdom is that as extinction intensity increases, the importance of biological traits range decrease and thus decreasing selectivity of extinction. My results indicate that this hypothesis is not entirely correct. Instead, I find that as baseline extinction risk increases, the effects of biological traits also increase. This is related to both the the correlation of the effects of biological traits and baseline extinction risk, and the correlation between the effects of biological traits.
  
  If accepted, all data and code necessary to duplicate this analysis will be made available on DRYAD.

  Possibly appropriate reviewers include Paul Harnik (paul.harnik@fandm.edu, Franklin and Marshall College), Steve Wang (scwang@swarthmore.edu, Swarthmore College), and Carl Simpson (simpsoncg@si.edu, Smithsonian Institution).
  
  Thank you for considering our work. Please send all correspondence regarding this manuscript to me via my email address (psmits@uchicago.edu).

  \closing{Sincerely,}

  \encl{Article; supplementary text, figures, tables.}

\end{letter}
\end{document}

