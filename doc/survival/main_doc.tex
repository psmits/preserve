\documentclass{article}

\usepackage{amsmath, amsthm}
\usepackage{setspace}
\usepackage{microtype, parskip}
\usepackage[comma,sort&compress]{natbib}
\usepackage{lineno}
\usepackage{docmute}
\usepackage{caption, subcaption, multirow, morefloats, rotating}
\usepackage{wrapfig}

\frenchspacing

\doublespacing

\raggedright

\begin{document}
\linenumbers
\modulolinenumbers[2]

%\maketitle

\begin{titlepage}
  \begin{large}
    \textbf{Title:} How do biological traits affect brachiopod taxonomic survival? A hierarchical Bayesian approach.
  \end{large}

  \textbf{Running title:} How do biological traits affect taxonomic survival?

  \textbf{Author:} Peter D Smits, psmits@uchicago.edu, Committee on Evolutionary Biology, University of Chicago

  \textbf{Keywords:} extinction, macroevolution, macroecology, Paleozoic, selection

  \textbf{Word count:} ?
  
  \textbf{Table count:} 0.
 
  \textbf{Figure count:} 5 main text, 3 supplement.

  \textbf{Data archival location:} If accepted, all data and code necessary to duplicate this analysis will be made available on DRYAD.

\end{titlepage}

\documentclass[12pt,letterpaper]{article}

\usepackage{amsmath, amsthm}
\usepackage{microtype, parskip}
\usepackage[comma,numbers,sort&compress]{natbib}
\usepackage{lineno}
\usepackage{docmute}
\usepackage{caption, subcaption, multirow, morefloats, rotating}
\usepackage{wrapfig}

\frenchspacing

\captionsetup[subfigure]{position = top, labelfont = bf, textfont = normalfont, singlelinecheck = off, justification = raggedright}

\begin{document}

\begin{abstract}
  How does ecology affect taxon duration? While the effect of geographic range on extinction risk is well documented, the effects of other traits are less well documented. Here, I analyze patterns of Paleozoic brachiopod genus durations and how various biological traits are related to systematic differences in expected extinction risk. I analyze geographic range, affinity for epicontinental seas versus open ocean environments, and body size. Additionally, I allow for environmental affinity to have a nonlinear effect on duration. I do this in a hierarchical Bayesian modeling context, allowing me to model the possible interaction between the effects of biological traits and time of origination. I find weak that as extinction risk increases, the expected strength of the selection gradient on biological traits (except body size) increases. This manifests as greater expected differences in extinction risk for each unit change in geographic range and environmental preference during periods of high extinction risk, while a much flatter expected selection gradient during periods of low extinction risk. I find evidence for a nonlinear relationship between environmental preference and extinction risk such that intermediate affinities (``generalists'') have a lower expected extinction risk than either end members (``specialists''). Interestingly, I find that as extinction risk increases, the peakedness of this relationship is expected to increases as well. These results demonstrate the importance of directly modeling the structure inherent in the observed data as a means to better understand which processes may have been driving the observed patterns of diversification.
\end{abstract}

\end{document}


\documentclass[12pt,letterpaper]{article}

\usepackage{amsmath, amsthm}
\usepackage{microtype, parskip}
\usepackage[comma,numbers,sort&compress]{natbib}
\usepackage{lineno}
\usepackage{docmute}
\usepackage{caption, subcaption, multirow, morefloats, rotating}
\usepackage{wrapfig}

\frenchspacing

\captionsetup[subfigure]{position = top, labelfont = bf, textfont = normalfont, singlelinecheck = off, justification = raggedright}

\begin{document}
\section{Introduction}

How do biological traits affect extinction risk? \citet{Jablonski1986} observed that during periods of high expected extinction risk, the effects of biological traits on survival decreased in importance. However, this pattern was weakest/absent in the effect of geographic range on survival \citep{Jablonski1986}. Biological traits are defined here as descriptors of a taxon's adaptive zone, which is the set of biotic--biotic and biotic--abiotic interactions that a taxon can experience. In effect, these are descriptors of a taxon's broad-sense ecology.

\citet{Jablonski1986} phrased their conclusions in terms of background versus mass extinction, but this scenario is readily transferable to a continuous variation framework as there is no obvious distinction in terms of extinction rate between these two states \citep{Wang2003}. I adopt a continuous variation framework as this is more amenable for modeling the relationship between taxon traits and extinction risk. Additionally, the \citet{Jablonski1986} scenario has strong model structure requirements in order to test its proposed macroevolutionary mechanism. Not only do the taxon trait effects need to be modeled, but the relationships between these effects need to be modeled as well. 

Two possible macroevolutionary mechanisms which may underly the pattern observed by \citet{Jablonski1986} are: the effect of geographic range on predictive survival remains constant and those of other biological traits decrease, and the effect of geographic range in predicting survival increases and those of other biological traits stay constant. 

I model taxon durations because trait based differences in extinction risk should manifest as differences in taxon durations. Namely, a species with a beneficial trait should survive longer, on average, than a species without that beneficial trait. Conceptually, taxon survival can be considered an aspect of ``taxon fitness'' along with expected lineage branching/origination rate \citep{Cooper1984,Palmer2012}. The analysis of taxon durations, or time from origination to extinction, falls under the purview of survival analysis, a field of applied statistics commonly used in health care \citep{Klein2003} but has a long history in paleontology \citep{Simpson1944,Simpson1953,VanValen1973,VanValen1979}.

Geographic range is widely considered the most important taxon trait for estimating differences in extinction risk at nearly all times with large geographic range associated with low extinction risk \citep{Jablonski1986,Jablonski1987,Jablonski2003,Payne2007}. I expect this to hold true nearly always.

% why does this matter?
%   environmental preference is hard to quantify
%     at macroecological levels, environment because biome (or macroenvironments)
%   epicontinental seas are no longer here
The primary environmental dichotomy observed in ancient marine systems is the contrast between epicontinental seas and open-ocean coastline environments \citep{Miller2009a,Peters2008,Sheehan2001b}. Epicontinental seas are a shallow-marine environment where, given a high enough sea-level, the ocean has spread over the surface of a continental shelf with a depth typically less than 100m. In contrast, open-ocean coastline environments have much greater variance in depth and do not cover the continental shelf. During the Paleozoic, epicontinental seas were widely spread globaly but declined over the Mesozoic and eventually disappeared during the Cenozoic as open-ocean coastlines became the dominant shallow-marine setting \citep{Peters2008,Miller2009a,Johnson1974}.

\citet{Miller2009a} demonstrated that during several mass extinctions taxa associated with open-ocean environments tend to have a greater extinction risk than those taxa associated with epicontinental seas. During periods of background extinction, however, they found no consistent difference between taxa favoring either environment. Because of this study, the following prediction for survival patterns can be made: as extinction risk increases, taxa associated with open-ocean environments should generally increase in extinction risk versus those that favor epicontinental seas.

There is also a possible nonlinear relationship between environmental preference and taxon duration. A long standing hypothesis is that generalists or unspecialized taxa will have greater survival than specialists \citep{Simpson1944,Liow2004a,Liow2007b,Nurnberg2013a,Nurnberg2015,Baumiller1993} \uppercase{Smits, in prep}. I allowed for environmental preference to possibly have a parabolic effect on species duration 

Body size, measured as shell length \citep{Payne2014}, was also considered as a potentially informative covariate. Body size is a proxy for metabolic activity and other, correlated, life history traits \citep{Payne2014}. There is no strong hypothesis of how body size effects extinction risk in brachiopods, meaning a positive, negative, or null effect are all pluasible.

I adopt a hierarchical Bayesian survival modeling approach, which represents a conceptual and statistical unification of the paleontological dynamic and cohort survival analytic approaches \citep{VanValen1973,VanValen1979,Raup1978,Raup1975,Foote1988,Baumiller1993,Simpson2006}. By using a Bayesian framework I am able to quantify the uncertainty inherent in the estimates of the effects of biological traits on survival, especially in cases where the covariates of interest (biological traits) are themselves known with error. 

\end{document}


\documentclass[12pt,letterpaper]{article}

\usepackage{amsmath, amsthm}
\usepackage{microtype, parskip}
\usepackage[comma,numbers,sort&compress]{natbib}
\usepackage{lineno}
\usepackage{docmute}
\usepackage{caption, subcaption, multirow, morefloats, rotating}
\usepackage{wrapfig}

\frenchspacing

\begin{document}
\section{Materials and Methods}

\end{document}


\documentclass[12pt,letterpaper]{article}

\usepackage{amsmath, amsthm}
\usepackage{graphicx,hyperref}
\usepackage{microtype, parskip}
\usepackage[comma,sort&compress]{natbib}
\usepackage{lineno}
\usepackage{docmute}
\usepackage{subcaption, multirow, morefloats}
\usepackage{wrapfig}

\frenchspacing

\captionsetup[subfigure]{position = top, labelfont = bf, textfont = normalfont, singlelinecheck = off, justification = raggedright}

\begin{document}
\section{Results}

% Model fit
% Model adequacy

% Model comparison


\end{document}


\documentclass[12pt,letterpaper]{article}

\usepackage{amsmath, amsthm}
\usepackage{microtype, parskip}
\usepackage[comma,numbers,sort&compress]{natbib}
\usepackage{lineno}
\usepackage{docmute}
\usepackage{caption, subcaption, multirow, morefloats, rotating}
\usepackage{wrapfig}

\frenchspacing

\captionsetup[subfigure]{position = top, labelfont = bf, textfont = normalfont, singlelinecheck = off, justification = raggedright}

\begin{document}
\section{Discussion}

% hypotheses
%   Jablonski1987 hypothesis
The results presented here are in an interesting combination of congruence and contrast with those of \citet{Jablonski1987}.  These results are partially consistent with \citet{Jablonski1987} because during periods of higher baseline extinction risk the effect of geographic range is greater than the effects of other taxon traits, though the macroevolutionary mechanism proposed by \citet{Jablonski1987} is not supported by my results. While \citet{Jablonski1987} hypothesized that, as extinction risk increases, the effect of taxon traits, except for the effect geographic range size, decrease in size. 

My results instead demonstrate that the only effect that has strong correlation with baseline extinction risk is that of geographic range (Fig. \ref{fig:corr}). Instead of the effects of other taxon traits decreasing with increasing baseline extinction risk, the effect of geographic range increases while the effects of other taxon traits have no strong correlation with baseline extinction risk. 

%   Miller2009a results
There are two mass extinction events are captured within the time frame considered here: the end Ordovician, and the end Devonian. The nature of the analysis used here means that the effect of the mass extinction is only captured with any clarity by the survival patterns of the origination cohort that immediately proceeds the event. These two cohorts are Hirnantian and Frasnian for the end Ordovician and end Devonian, respectively.

For the Hirnantian, there is a 57\% posterior probability that taxa which tend to favor epicontinental seas are expected to survive longer than those favoring open ocean environments. Similarly, for the Frasnian there is a 47\% posterior probability that taxa which tend to favor epicontinental seas are expected to survive longer than those favoring open ocean environments. These probabilities were calculated as the percent of posterior draws of \(f(v_{i})\) which have their optima greater than 0, or towards favoring epicontinental. 

These results demonstrate no support for the observation of \citet{Miller2009a} that epicontinental seas are favored at mass extinction boundaries. These results, also, do not support the opposite conclusion. Instead, they are equitable. However, this analysis may not be capturing the patterns analyzed in \citet{Miller2009a} as my focus is on long term taxonomic duration between and within origination cohorts and not patterns of instantaenous extinction rates between stages. Given both of these factors, it is hard to make strong conclusions about the results of \citet{Miller2009a}.  
%   generalists versus specialists
There is an approximate 75\% posterior probability that taxa with intermediate environmental preferences have a lesser extinction risk than either end members, though the over all curvature of \(f(v_{i}\) is not very peaked meaning that this relationship does not lead to very strong differences in extinction risk (Fig. \ref{fig:env_mean}). This result supports the hypothesis that, in general, environmental generalists survive greater than environmental specialists \citep{Simpson1944,Liow2004a,Liow2007b,Nurnberg2013a,Nurnberg2015}.

The variance in estimate of the overall \(f(v_{i}\) reflects the large between cohort variance in cohort specific estimates of \(f(v_{i})\) (Fig. \ref{fig:env_cohort}). Given that there is only a 75\% posterior probability that the expected overall estimate of \(f(v_{i})\), it is not surprising that there are some stages where the theorized relationship is in fact reversed. Additionally, as discussed earlier, there are some stages where \(f(v_{i})\) does not resemble the theorized nonlinear relation with the optimum in the middle but instead is highly skewed or effectively linear (Fig. \ref{fig:env_cohort}). 

These results do not refute ``survival of the unspecialized'' as a time-invariant generalization, but instead demonstrate how while the expected group-level estimate of \(f(v_{i})\) might favor one hypothesis there is still enough variability between cohorts so that in some realizations this pattern may not hold or can even be reversed. These results are also consistent with aspects of \citep{Miller2009a} who found that the effect of environmental preference on extinction risk was quite variable and without obvious patterning during times of background extinction.

% defense
%   species:genus?
%   difficulty towards tails, but that's to be expected
%     this model is about expectations, not tails/extreme events
%     this model is ok for the main part of the data
%     though, of course, this model has a long way to go (all models are false)

% future direction
%   other measures of ecology? affixing strategy a la Alexander1977
%   integration of phylogenetic information/taxonomic component
%     see Smits Submitted
%   comparison with other major groups in hierarchical model

% concluding statements
%   different mechanism than proposed by Jablonski1986, but same overall observation
%   no conclusive support for either Miller2009a or its converse
%   support for survival of the unspecialized

\end{document}


\section*{Acknowledgements}
I would like to thank K. Angielzcyk, M. Foote, P. D. Polly, and R. Ree for helpful discussion and advice. Additionally, thank you A. Miller for the epicontinental versus open-ocean assignments. This entire study would would not have been possible without the Herculean effort of the many contributors to the Paleobiology Database. In particular, I would like to thank J. Alroy, M. Aberhan, D. Bottjer, M. Clapham, F. F\"{u}rsich, N. Heim, A. Hendy, S. Holland, L. Ivany, W. Kiessling, B. Kr\"{o}ger, A. McGowan, T. Olszewski, P. Novack-Gottshall, M. Patzkowsky, M. Uhen, L. Villier, and P. Wager. This work was supported by a NASA Exobiology grant (NNX10AQ446) to A. Miller and M. Foote. This is Paleobiology Database publication XXX.

\clearpage

\bibliographystyle{evolution}
\bibliography{newbib,packages}

\clearpage

\appendix
\input{appendix}

\end{document}
