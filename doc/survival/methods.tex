\documentclass[12pt,letterpaper]{article}

\usepackage{amsmath, amsthm}
\usepackage{microtype, parskip}
\usepackage[comma,numbers,sort&compress]{natbib}
\usepackage{lineno}
\usepackage{docmute}
\usepackage{caption, subcaption, multirow, morefloats, rotating}
\usepackage{wrapfig}

\frenchspacing

\begin{document}
\section{Materials and Methods}

\subsection{Analytical approach}

Hierarchical modelling, sometimes called ``mixed-effects modeling,'' is a statistical approach which explicitly takes into account the structure of the observed data in order to model both the within and between group variance \citep{Gelman2013d,Gelman2007}. The units of study (e.g. genera) each belong to a single grouping (e.g. origination cohort). These groups are considered draws from a shared probability distribution (e.g. all cohorts, observed and unobserved). The group-level parameters are then estimated simultaneously as the other parameters of interest (e.g. covariate effects) \citep{Gelman2013d}. The subsequent estimates are partially pooled together, where parameters from groups with large samples or effects remain large while those of groups with small samples or effects are pulled towards the overall group mean. 

This partial pooling is one of the greatest advantages of hierarchical modeling. By letting the groups ``support'' each other, parameter estimates then better reflect our statistical uncertainty. Additionally, this partial pooling helps control for multiple comparisons and possibly spurious results as effects with little support are drawn towards the overall group mean \citep{Gelman2013d,Gelman2007}. 

All covariate effects (regression coefficients), as well as the intercept term (baseline extinction risk), were allowed to vary by group (origination cohort). The covariance/correlation between covariate effects was also modeled. This hierarchical structure allows inference for how covariates effects may change with respect to each other while simultaneously estimating the effects themselves, propagating our uncertainty through all estimates. 

Additionally, instead of relying on point estimates of environmental affinity, I treat environmental affinity as a continuous measure of the difference between the taxon's environmental occurrence pattern and the background occurrence pattern (Appendix \ref{sec:uncer}).


\subsection{Fossil occurrence information}

The dataset analyzed here is derived from the a combination of the occurrence information from \citet{Foote2013} and the body size data from \citet{Payne2014}. The \citet{Foote2013} dataset is based on the Paleobiology Database (http://www.paleodb.org); see \citet{Foote2013} for a full description of the inclusion criterion. Additionally, epicontinental versus open ocean assignemnts for occurrence information are based on \citet{Miller2009a}. \uppercase{note: I don't know how this may need to be updated.}

Sampled occurrences were restricted to those with paleolatitude and paleolongitude coordinates, assignment to either epicontinental or open-ocean environment, and belonging to a genus present in the body size dataset \citep{Payne2014}. Genus duration was calculated as the number of geologic stages from first appearance to last appearance, inclusive. Genera with a last occurrence in or after Changhsingian stage were right censored at the Changhsingian. Genera with a duration of only one stage were left censored (Appendix \ref{sec:cen}). The covariates used to model genus duration were geographic range size (\(r\)), environmental preference (\(v, v^{2}\)), and body size (\(m\)). 

Geographic range was calculated using an occupancy approach. First, all occurrences were projected onto an equal-area cylindrical map projection. Each occurrence was then assigned to one of the cells from a 70 \(\times\) 34 regular raster grid placed on the map. Each grid cell represents approximately 250,000 km\(^{2}\). The map projection and regular lattice were made using shape files from http://www.naturalearthdata.com/ and the \texttt{raster} package for R \citep{raster}.

For each stage, the total number of occupied grid cells, or cells in which a fossil occurrs, was calculated. Then, for each genus, the number of grid cells occupied by that genus was calculated. Dividing the genus occupancy by the total occupancy gives the relative occupancy of that genus. Mean relative genus occupancy was then calculated as the mean of the per stage relative occupancies of that genus. 

Body size data was sourced directly from \citet{Payne2014}. Because those measurements are presented without error, a measurement error model similar to the one for environmental affinity could not be implemented (Appendix \ref{sec:uncer}).

Prior to analysis, geographic range and body size were transformed and standardized in order to improve interpretability of the results. Geographic range, which can only vary between 0 and 1, was logit transformed. Body size, which is defined for all positive real values, was natural log transformed. These covariates were then standardized by mean centering and dividing by two times their standard deviation following \citet{Gelman2007}.


\subsection{Survival model}
Genus durations were modeled as time-till-event data \citep{Klein2003}, with covariate information used in estimates of extinction risk as a hierarchical regression model. Genus durations were assumed to follow either an exponential or Weibull distribution. Each of these distributions makes assumptions about how duration may effect extinction risk \citep{Klein2003}. The exponential distribution assumes that extinction risk is independent of duration. In contrast, the Weibull distribution allows for age dependent extinction via the shape parameter \(\alpha\), though only as a monotonic function of duration. Importantly, the Weibull distribution is equivalent to the exponential distribution when \(\alpha = 1\). 

The following variables are defined: \(y_{i}\) is the duration of genus \(i\) in geologic stages, \(X\) is the matrix of covariates including a constant term, \(B_{j}\) is the vector of regression coefficients for origination cohort \(j\), \(\Sigma\) is the covariance matrix of the regression coefficients, \(\tau\) is the vector of scales the standard deviations of the between-cohort variation in regression coefficient estimates, and \(\Omega\) is the correlation matrix of the regression coefficients.

The exponential model is defined
\begin{equation}
  \begin{aligned}
    y_{i} &\sim \mathrm{Exponential}(\lambda) \\
    \lambda_{i} &= \exp(\mathbf{X}_{i} B_{j[i]}) \\
    B &\sim \mathrm{MVN}(\vec{\mu}, \Sigma) \\
    \Sigma &= \text{Diag}(\vec{\tau}) \Omega \text{Diag}(\vec{\tau}) \\
    \mu_{\kappa} &\sim 
    \begin{cases} 
      \mathcal{N}(0, 5) & \text{if } k \neq r, or \\
      \mathcal{N}(-1, 1) & \text{if } k = r \\
    \end{cases} \\
    \tau_{\kappa} &\sim \mathrm{C^{+}}(1) \text{ for } \kappa \in 1:k \\
    \Omega &\sim \text{LKJ}(2).
  \end{aligned}
  \label{eq:exp_total}
\end{equation}

Similarly, the Weibull model is defined
\begin{equation}
  \begin{aligned}
    y_{i} &\sim \mathrm{Weibull}(\alpha, \sigma) \\
    \sigma_{i} &= \exp\left(\frac{-(\mathbf{X}_{i} B_{j[i]})}{\alpha}\right) \\
    B &\sim \mathrm{MVN}(\vec{\mu}, \Sigma) \\
    \Sigma &= \text{Diag}(\vec{\tau}) \Omega \text{Diag}(\vec{\tau}) \\
    \alpha &\sim \mathrm{C^{+}}(2) \\
    \mu_{k} &\sim 
    \begin{cases} 
      \mathcal{N}(0, 5) & \text{if } k \neq r, or \\
      \mathcal{N}(-1, 1) & \text{if } k = r \\ 
    \end{cases} \\
    \tau_{k} &\sim \mathrm{C^{+}}(1) \\
    \Omega &\sim \text{LKJ}(2).
  \end{aligned}
  \label{eq:wei_total}
\end{equation}
The principle difference between this model and the previous (Eq. \ref{eq:exp_total}) is the inclusion of the shape parameter \(\alpha\). Note that \(\sigma\) is approximately equivalent to \(1 / \lambda\).

For an explanation of how this model was developed and parameter explanations, please see Appendix \ref{sec:survival}. Note that these models (Eq. \ref{eq:exp_total}, \ref{eq:wei_total}) do not include how the uncertainty in environmental affinity is included nor how censored observations are included. For an explanation of both of these aspects, see Appendices \ref{sec:uncer} and \ref{sec:cen}.

\subsection{Parameter estimation}
The  joint posterior was approximated using a Markov-chain Monte Carlo routine that is a variant of Hamiltonian Monte Carlo called the No-U-Turn Sampler \citep{Hoffman2014} as implemented in the probabilistic programming language Stan \citep{2014stan}. The posterior distribution was approximated from four parallel chains run for 10,000 draws each, split half warm-up and half sampling and thinned to every 10th sample for a total of 5000 posterior samples. Chain convergence was assessed via the scale reduction factor \(\hat{R}\) where values close to 1 (\(\hat{R} < 1.1\)) indicate approximate convergence. Convergence means that the chains are approximately stationary and the samples are well mixed \citep{Gelman2013d}.


\subsection{Model evaluation}

Models were evaluated using both posterior predictive checks and an estimate of out-of-sample predictive accuracy. The motivation behind posterior predictive checks as tools for determining model adequacy is that replicated data sets using the fitted model should be similar to the original data \citep{Gelman2013d}. Systematic differences between the simulations and observations indicate weaknesses of the model fit. An example of a technique that is very similar would be inspecting the residuals from a linear regression.

The strategy behind posterior predictive checks is to draw simulated values from the joint posterior predictive distribution, \(p(y^{rep} | y)\), and then compare those draws to the empirically observed values \citep{Gelman2013d}. To accomplish this, for each replicate, a single value is drawn from the marginal posterior distributions of each regression coefficient from the final model as well as \(\alpha\) for the Weibull model (Eq. \ref{eq:exp_total}, \ref{eq:wei_total}). Then, given the covariate information \(\mathbf{X}\), a new set of \(n\) genus durations are generated giving a single replicated data set \(y^{rep}\). This is repeated 1000 times in order to provide a distribution of possible values that could have been observed given the model. 

In order to compare the fitted model to the observed data, various graphical comparisons or test quantities need to be defined. The principal comparison used here is a comparison between non-parameteric approximation of the survival function \(S(t)\) as estimated from both the observed data and each of the replicated data sets. The purpose of this comparison is to determine if the model approximates the same survival/extinction pattern as the original data. 

The exponential and Weibull models were compared for out-of-sample predictive accuracy using the widely-applicable information criterion (WAIC) \citep{Watanabe2010a}. Out-of-sample predictive accuracy is a measure of the expected fit of the model to new data. However, because the Weibull model reduces to the exponential model when \(\alpha = 1\) my interest is not in choosing between these models. Instead, comparisons of WAIC values are useful for better understanding the effect of model complexity on out-of-sample predictive accuracy. The calculation of WAIC used here corresponds to the ``WAIC 2'' formulation recommended by \citet{Gelman2013d}. For an explanation of how WAIC is calculated, see Appendix \ref{sec:waic}. Lower values of WAIC indicate greater expected out-of-sample predictive accuracy than higher values.

\end{document}
